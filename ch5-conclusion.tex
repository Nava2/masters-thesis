\chapter{Conclusion}
\label{ch:conclusion}

% Yeah, I think we all know what goes here.

\section{Contributions}
\label{sec:contributions}

\section{Future Work}
\label{sec:future-work}

\begin{enumerate}
    \item Incorporating data collection from the event-loop for processing in software like modelsim
\end{enumerate}

\section{Recommendations}

\subsubsection{Design considerations\cite{Nakamura2013}}

Utilizes a simple dual-BUS architecture for address and data. Uses memory-mapped ``I/O space'' controller for a port-mapped I/O scheme. An interrupt controller is built consisting of a single register, \verb|intr| that stores information about what device interrupted. The authors used a simplified interrupt model in which only one interrupt is supported at a given time. The addition of the interrupt support required the addition of ``RETURN'' and ``CALL'' instructions to support subroutines. Any interrupt implementation must have these machine instructions specified to give the controller the ability to change execution flow whilst maintaining state of the machine. 

\section{Next steps}

\begin{enumerate}
    \item high-level is awesome
    \item Lua is fucking awesome
    \item Sol is fucking awesome
    \item Sol makes compilers sad
    \item Being too fancy with high-level languages creates issues when teaching low-level concepts that leads to ambiguities in teaching ideas
    \item C++ development is insanely slow compared to non-machine languages
\end{enumerate}