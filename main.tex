%%%%%%%%%%%%%%%%%%%%%%%%%%%%%%%%%%%%%%%%%%%%%%%%%%%%%%%%%%%%%%%%%%%%%%%%%%%%%%%%
% University of Western Ontario Thesis Template
% By: Justin Quinn Veenstra, 2010
% With thanks to Mr. (soon to be Dr.) Will Robertson.


\documentclass[12pt, twoside, final]{report}
%% Decomment next line to use PostScript fonts
%%\UsePackage{times}
%%%%%%%%%%%%%%%%%%%%%%%%%%%%%%%%%%%%%%%%%%%%%%%%%%%%%%%%%%%%%%%%%%%%%%%%
%%                                                                    %%
%%                    ***   I M P O R T A N T   ***                   %%
%%                                                                    %%
%% Fill in the following fields with the required information:        %%
%%  - \department{...}  name of the graduate department               %%
%%  - \degree{...}      name of the degree obtained                   %%
%%  - \author{...}      name of the author                            %%
%%  - \title{...}       title of the thesis                           %%
%%  - \gyear{...}       year of graduation                            %%
%%  - \super{...}    supervisor
%%  - \firstname, \middlename, \lastname... there is additional documentation by the actual fields, so I'll leave it at that
%%%%%%%%%%%%%%%%%%%%%%%%%%%%%%%%%%%%%%%%%%%%%%%%%%%%%%%%%%%%%%%%%%%%%%%%
\usepackage{appendix}
\usepackage{graphicx}
\usepackage{amsmath}
\usepackage[byname]{smartref}
\usepackage{hyperref} % comment out for hardcopy, hyperref must be loaded before cleveref
\usepackage{txfonts}
\usepackage{tocloft}

\usepackage[doublespacing]{setspace}

% Topic listings that are conditional to draft
\usepackage[obeyFinal]{todonotes}

\usepackage{subcaption}

\usepackage[inline]{enumitem} % load before cleveref

\usepackage{lipsum}
\usepackage[backend=biber, style=ieee]{biblatex}

\addbibresource{online.bib}
\addbibresource{print.bib}

\usepackage[chapter]{minted}
\setminted{linenos, frame=single, autogobble=true}

\newcounter{sublisting}[listing]

\usepackage{etoolbox}
\BeforeBeginEnvironment{minted}{\begin{singlespace}}
\AfterEndEnvironment{minted}{\end{singlespace}}
\BeforeBeginEnvironment{displayquote}{\begin{spacing}{1.5}}
\AfterEndEnvironment{displayquote}{\end{spacing}}
\BeforeBeginEnvironment{textquote}{\begin{spacing}{1.5}}
\AfterEndEnvironment{textquote}{\end{spacing}}

% MUST be defined after {minted}
\usepackage{algorithm, algpseudocode}

\usepackage{comment}

\usepackage{csquotes}

\usepackage{makecell}

\usepackage{array,multirow}

\if@final%
\else%
%    \usepackage[pass,showframe]{geometry}
\fi

% Always use capitalizaion, include the name and do not abbreviate the 
% label type, this is similar to how \\autoref works. 
\usepackage[capitalize, nameinlink, noabbrev]{cleveref}

% Use for doing total calculations, handy to not have to update stuff later :)
\usepackage{totcount}
\regtotcounter{chapter}

% Converts numbers to words!
\usepackage{fmtcount}

\interfootnotelinepenalty=10000 

\makeatletter
\numberwithin{figure}{chapter}

\newenvironment{dedication}%
{\clearemptydoublepage 
 \begin{center}
  \section*{Dedication}
 \end{center}
 \begingroup
}{\newpage\endgroup}

\newenvironment{preliminary}%
{\pagestyle{plain}\pagenumbering{roman}}%
{\pagenumbering{arabic}}

\addtoreflist{chapter}
\newtheorem{theorem}{Theorem}[section]
\newtheorem{lemma}[theorem]{Lemma}
\newtheorem{proposition}[theorem]{Proposition}
\newtheorem{corollary}[theorem]{Corollary}

\newenvironment{proof}[1][Proof]{\begin{trivlist}
\item[\hskip \labelsep {\bfseries #1}]}{\end{trivlist}}
\newenvironment{definition}[1][Definition]{\begin{trivlist}
\item[\hskip \labelsep {\bfseries #1}]}{\end{trivlist}}
\newenvironment{example}[1][Example]{\begin{trivlist}
\item[\hskip \labelsep {\bfseries #1}]}{\end{trivlist}}
\newenvironment{remark}[1][Remark]{\begin{trivlist}
\item[\hskip \labelsep {\bfseries #1}]}{\end{trivlist}}

\newcommand{\qed}{\nobreak \ifvmode \relax \else
      \ifdim\lastskip<1.5em \hskip-\lastskip
      \hskip1.5em plus0em minus0.5em \fi \nobreak
      \vrule height0.75em width0.5em depth0.25em\fi}

% Default values for title page.

%% To produce output with the desired line spacing, the argument of
%% \spacing should be multiplied by 5/6 = 0.8333, so that 1 1/2 spaced
%% corresponds to \spacing{1.5} and double spaced is \spacing{1.66}.
\def\normalspacing{1.25} % default line spacing

%% Define the "thesis" page style.
\if@twoside % If two-sided printing.
\def\ps@thesis{\let\@mkboth\markboth
   \def\@oddfoot{}
   \let\@evenfoot\@oddfoot
   \def\@oddhead{
      {\sc\rightmark} \hfil \rm\thepage
      }
   \def\@evenhead{
      \rm\thepage \hfil {\sc\leftmark}
      }
   \def\chaptermark##1{\markboth{\ifnum \c@secnumdepth >\m@ne
      Chapter\ \thechapter. \ \fi ##1}{}}
   \def\sectionmark##1{\markright{\ifnum \c@secnumdepth >\z@
      \thesection. \ \fi ##1}}}
\else % If one-sided printing.
\def\ps@thesis{\let\@mkboth\markboth
   \def\@oddfoot{}
   \def\@oddhead{
      {\sc\rightmark} \hfil \rm\thepage
      }
   \def\chaptermark##1{\markright{\ifnum \c@secnumdepth >\m@ne
      Chapter\ \thechapter. \ \fi ##1}}}
\fi

\pagestyle{thesis}
% Set up page layout.
\setlength{\textheight}{9in} % Height of the main body of the text
\setlength{\topmargin}{-.5in} % .5" margin on top of page
\setlength{\headsep}{.5in}  % space between header and top of body
\addtolength{\headsep}{-\headheight} % See The LaTeX Companion, p 85
\setlength{\footskip}{.5in}  % space between footer and bottom of body
\setlength{\textwidth}{6.25in} % width of the body of the text
\setlength{\oddsidemargin}{.25in} % 1.25" margin on the left for odd pages
\setlength{\evensidemargin}{0in} % 1.25"  margin on the right for even pages

% Marginal notes
\setlength{\marginparwidth}{.75in} % width of marginal notes
\setlength{\marginparsep}{.125in} % space between marginal notes and text

% Make each page fill up the entire page. comment this out if you
% prefer. 
\flushbottom

\setcounter{tocdepth}{3} % Number the subsubsections 
\def\normalspacing{2.0} % default line spacing

\newcommand\acknowlege[1]{%
  \edef\@tempa{#1}%
  \def\@tempb{}%
  \ifx\@tempa\@tempb
  \else%
    \newpage%
    \Large\begin{center}\textbf{Acknowledgements}\end{center}%
    \\\normalsize%
    \indent\\#1%
    \newpage%
  \fi
}

%\renewcommand{\appendixtocname}{\Huge \textbf{List of Appendices} \normalsize}
\newcommand{\blank}{\hspace{-2mm}}
\newcommand{\super}{Dr. K. McIsaac} %supervisor
\newcommand{\superj}{} %joint supervisor, if there is one, leave blank if not (lbin)... only one of the three.
\newcommand{\superc}{} %co-supervisor, if there is one, leave blank if not (lbin)
\newcommand{\supera}{} %alternate supervisor, if there is one, leave blank if not (lbin)
\newcommand{\sco}{Dr. W. J. Braun}  %member of supervisory committee
\newcommand{\sct}{Dr. A. Bing}  %other member of supervisory committee (lbin)
\newcommand{\examo}{Dr. Q. Ring}  %examining committee (up to four, if less leave blank)
\newcommand{\examt}{Dr. W. Fing}
\newcommand{\examth}{Dr. G. Hing}
\newcommand{\examf}{}
\newcommand{\department}{Electrical and Computer Engineering}
\newcommand{\degree}{Masters of Engineering Science}
\newcommand{\firstname}{Kevin}
\newcommand{\middlename}{A}
\newcommand{\lastname}{Brightwell}
%\renewcommand{\author}[1]{\ifx\empty#1\else\gdef\@author{#1}\fi} 
\newcommand{\authorname}{{\firstname} {\middlename} {\lastname}}
\newcommand{\titl}{Improvements in Microprocessor Engineering Education Through Simulation Technologies}
\newcommand{\spinetitle}{Microprocessor Engineering Education via Simulation}%only if the above is more than 60 characters
\newcommand{\thesisformat}{Monograph} %or Integrated Article
\newcommand{\gyear}{\number\year}
\newcommand{\makecoauthor}{
%Type information about coauthorship here/

}
\newcommand{\makeacknowlege}{%

I would like to acknowledge my advisor Dr. Kenneth McIsaac for his continued support and mentorship throughout my undergraduate and graduate works. Thank you for taking the time to debug, construct, and destruct ideas from all angles. Your direction has allowed me to grow as a research, engineer and academic. 

I would like to thank my peers in Dr. McIsaac's lab for providing laughs and support over the course of this work. I would like to extend the same thanks to those in Dr. Xianbin Wang and Dr. Roy Eagleson's labs for being supportive peers through forced geographic association.  

I would like to thank my father and step mother, John and Amy Brightwell, and my mother and step father, Kyra Hanson and Wilf Wityshyn, for listening to my complaints, failures and successes throughout this work and never batting an eye of support. I would like to thank Mr. Brad de Vlugt, Mr. Kyle Fricke, and Ms. Trinette Wright for their continued support over the course of this work providing friendship and a talking post repeatedly. Lastly, I would like to thank my boyfriend Eric Pattara for his love, continuous support and delightful immaturity when I needed that most. 

To any and all of my colleagues, peers, family and friends thank you for your support.
}
\newcommand{\listappendixname}{List of Appendices}
\newlistof{myappendices}{app}{\listappendixname}
\newcommand{\myappendices}[1]{%
\addcontentsline{app}{myappendices}{#1}\par}

\linespread{2}

%% ***   NOTE   ***
%% You should put all of your '\newcommand', '\newenvironment', and
%% '\newtheorem's (in other words, all the global definitions that you
%% will need throughout your thesis) in a separate file and use
%% "\input{filename}" to input it here.



\makeatother
\begin{document}

%% This sets the page style and numbering for preliminary sections.
\begin{preliminary}
\begin{spacing}{1.5}

%% This generates the title page from the information given above.
%\maketitle
%\addcontentsline{toc}{chapter}{Certificate of Examination}
%\makecert
%\newpage
%\addcontentsline{toc}{chapter}{Co-Authorship Statement}
%\coauthor{\makecoauthor}  %comment this out if none
%\newpage

\phantomsection
\addcontentsline{toc}{chapter}{Abstract}
\Large\begin{center}\textbf{Abstract}\end{center}\normalsize
%%  ***  Put your Abstract here.   ***
%% (150 words for M.Sc. and 350 words for Ph.D.)

In this thesis, we aim to improve the outcomes of students learning Computer Architecture and Embedded Systems topics within Software and Computer Engineering programs. We develop a simulation of processors that attempts to improve the visibility of hardware within the simulation environment and replace existing solutions in use within the classroom. We designate a series of requirements of a successful simulation suite based on current state-of-the-art simulations within literature. Provided these requirements, we build a quantitative rating of  the same set of simulations. Additionally, we rate our previously implemented tool, hc12sim, with current solutions. Using the gaps in implementations from our state-of-the-art survey, we develop two solutions. First, we developed a web-based solution using the Scala.js compiler for Scala with an event-driven simulation engine through Akka. This Scala model implements a VHDL-like DSL for instruction control definition. Next we propose tools for developing cross-platform native applications through a project-based build system within CMake and a continuous integration pipeline using Vagrant, Oracle VirtualBox and Jenkins. Lastly, we propose a configuration-driven processor simulation built from the original hc12sim project that utilizes a Lua-based scripting interface for processor configuration. While we considered other high-level languages, Lua best fit our requirements allowing students to use a modern high-level programming language for processor configuration. Instruction controls are defined through Lua functions using high-level constructs that implicitly trigger low-level simulation events. Lastly, we conclude with suggestions for building a new solution that would better meet requirements set forth in our research question building from successful aspects from this work. 

\vfill
\textbf{Keywords:} Embedded systems, computer architecture, computer architecture simulation, pedagogy, cross-platform application development
\newpage

\phantomsection
\addcontentsline{toc}{chapter}{Acknowledgements}
\Large\begin{center}\textbf{Acknowledgements}\end{center}\normalsize
\makeacknowlege
\clearpage 

\tableofcontents
\newpage
\newpage

\phantomsection
\addcontentsline{toc}{chapter}{List of Figures}
\listoffigures
\newpage

\phantomsection
\addcontentsline{toc}{chapter}{List of Tables}
\listoftables\newpage

\phantomsection
\addcontentsline{toc}{chapter}{List of Listings}
\listoflistings\newpage

\phantomsection
\addcontentsline{toc}{chapter}{List of Definitions}

\chapter*{List of Definitions}

\begin{definition}[Embedded Systems] 
	Topics involving characteristics of embedded systems, techniques for embedded applications, parallel input and output, synchronous and asynchronous serial communication, interrupt handling, applications involving data acquisition, control, sensors, and actuators, implementation strategies for complex embedded systems \cite[p.~118]{cec2016}
\end{definition}

\begin{definition}[Computer Architectures] 
	Topics involving computer organization and architecture. Including, but not limited to processor organization, instruction set architecture, memory system organization, performance, and interfacing fundamentals\cite[p.~118]{cec2016}
\end{definition}

\begin{definition}[Instruction Set Architecture (ISA)]
    The design of an interface between software and hardware designs in a computing environment. Including but not limited to assembly language and microcode design. Once realized, an ISA is referred to as an Instruction Set Implementation. 
\end{definition}

\begin{definition}[Reduced Instruction Set Computer (RISC)]
    A type of microprocessor architecture that utilizes small, highly-optimized set of instructions used to reduce power and improve instruction performance\cite{Aletan1992, Stokes1999}. These architectures are commonly used in embedded, mobile (ARM), and high-performance server applications (IBM POWER). 
\end{definition}

\begin{definition}[Complex Instruction Set Computer (CISC)]
    A type of microprocessor architecture that utilizes larger, instructions aiming to reduce the software instructions required to complete a task\cite{Aletan1992, Stokes1999}. CISC architectures are used in most general applications based on Intel\textregistered{} x86 ISA\cite{intel2017}.
\end{definition}

\begin{definition}[Integrated Development Environment (IDE)]
    A tool used for developing hardware and software components that provides integrated tools for the current development context. For example, Xilinx ISE\cite{xilinxISE}, JetBrains IntelliJ IDEA\footnote{Available: \url{https://www.jetbrains.com/idea/}}.
\end{definition}

\begin{definition}[Hardware Description Language (HDL)]
    A programming language used to describe the structure and behaviour of electronic circuits. These languages are used to synthesize simulated and physical circuit layout from the gate and register transfer level\cite{Chu2006}.
\end{definition}

\begin{definition}[Graphical User Interface (GUI)]
    Any interface based on rendering graphically, for example a window with buttons. 
\end{definition}

\begin{definition}[Arithmetic Logic Unit (ALU)]
    In a computer system, ALUs are responsible for performing arithmetic operations on binary input such as addition, shifts, multiplication, division.
\end{definition}

\begin{definition}[Domain-Specific Language (DSL)]
    A computer programming language that is specialized to a particular application space (domain). For example, HTML is a DSL used to layout websites but could not be used to program an embedded device. Similarly, C++ is an excellent language for creating applications but is very poor at describing layout for a webpage.
\end{definition}

\begin{definition}[Transpiler]
    A compiler that compiles one language to another. Typical transpilers convert one language to JavaScript to allow use in web applications. For example, Scala.js and TypeScript both compile their respective language to JavaScript.
\end{definition}

\begin{definition}[Reification]
    The process by which an abstraction (usually a \javainline{String} representation) of software is converted into an explicit object. For example, JavaScript's \jsinline{eval(script)} compiles and executes the content of the \verb|script| passed and any state changes within the script are applied to the global scope.
\end{definition}

\begin{definition}[Just-in-Time Compiler (JIT)]
    A traditional compiler that operates at runtime to perform optimization on running code. JITs are typically found in interpreted languages such as on the JVM or in JavaScript. JITs vastly improve the runtime of the code they compile as they often utilize semantics such as code path tracing and known constant analysis that is not necessarily available at compile-time. 
\end{definition}

\phantomsection
\addcontentsline{toc}{chapter}{List of Appendices}
\listofmyappendices\newpage
%\addcontentsline{toc}{chapter}{List of Abbreviations, Symbols, and Nomenclature}
%\large List of Abbreviations, Symbols, and Nomenclature \normalsize
%\newpage

\if@final%
\else%
\todototoc
\listoftodos
\fi
\end{spacing}
\end{preliminary}
%% End of the preliminary sections: reset page style and numbering.

%%%%%%%%%%%%%%%%%%%%%%%%%%%%%%%%%%%%%%%%%%%%%%%%%%%%%%%%%%%%%%%%%%%%%%%%
%%                                                                    %%
%%                    ***   I M P O R T A N T   ***                   %%
%%                                                                    %%
%% Put your Chapters here; the easiest way to do this is to keep each %%
%% chapter in a separate file and \include all the files right here.  %%
%% Note that each chapter file should start with the line             %%
%% "\chapter{ChapterName}".  Note that using "\include" instead of    %%
%% "\input" makes each chapter start on a new page.                   %%
%%%%%%%%%%%%%%%%%%%%%%%%%%%%%%%%%%%%%%%%%%%%%%%%%%%%%%%%%%%%%%%%%%%%%%%%

\chapter{Introduction}
\label{ch:introduction}

\begin{comment}
\begin{itemize}

\item What research question(s) are you asking?
\begin{itemize}
\item Can improvements to software simulation stories improve learning outcomes for students in electrical, computer and software engineering on topics surrounding microcontrollers? 
\item Do familiar high-level software constructs improve student learning outcomes of low-level hardware components?
\item What are the requirements needed for a simulation suite to be successful in a pedagogical space rather than industrial?
\item What curriculum changes are required to improve student outcomes?
\end{itemize}

\item Why are you asking it/them?
\begin{itemize}
\item Drop in marks for students within these courses
\item Students voice frustration with course material and current solutions
\end{itemize}

\item Has anyone else done anything similar?
\begin{itemize}
\item Yes. 
\end{itemize}
\item Is your research relevant to research/practice/theory in your field?
\begin{itemize}
\item Education is always required
\item Improvements are required as IoT increases and devices are moving back towards small
microelectronics
\end{itemize}
\item What is already known or understood about this topic?
\begin{itemize}
\item Simulations: \cite{Tappan2009}, \cite{Skrien2001}, \cite{Skillen2011}
\item Curriculum: 
\end{itemize}
\item How might your research add to this understanding, or challenge existing theories and beliefs?
\begin{itemize}
\item Extend on existing solution
\item Propose opportunity to replace multiple configurations, not a single one
\end{itemize}
\end{itemize}
\end{comment}

Embedded systems and computer architectures are a critical part of both computer and software engineering undergraduate programs \cite{cec2016, sec2015, ece-ce-program, Ristov2011, Stolikj2011}. Over time, it is expected that knowledge of embedded systems and computer architectures will be required given the rapid growth of positions in both systems software developers (+13\%) and computer occupations (+12.5\%) \cite{bls2014}. At the author's institution, \uwo{}, two courses cover embedded systems and computer architectures, ECE 3375 - Microprocessors and Microcomputers and ECE 3390 - Hardware/Software Co-Design \cite{eceOutlines}. These topics are taught through the use of industry based software such as Intel Quartus Prime (formerly Altera Quartus II) \cite{quartus}, Xilinx ISE WebPACK \cite{xilinxISE} or WinIDE \cite{winide}. These established industry tools are complex and powerful for industry but often have students feeling overwhelmed under use. Additionally, with the prevalence of high-level programming languages, embedded systems work has become increasingly difficult for students to connect high-level concepts to low-level details in computer architectures largely due to theoretical discontinuities between lecture materials and the ``black box'' of hardware. 

\section{Research Question}

This thesis attempts to answer the following research question:
\begin{displayquote}
Can improvements and increased usage of software simulation technologies within laboratory exercises in undergraduate embedded system and computer architecture courses improve student engagement in laboratory and assignment exercises?
\end{displayquote}
Given the overall research question, there exist several sub-questions: 
\begin{enumerate}
	\item Are simulation software considered beneficial to student outcomes in computer and software engineering programs?
	\item What elements of simulation software are required for successful implementation within an undergraduate course?
\end{enumerate}
Each of these questions are elaborated in further sections. 

\section{Motivation}
\label{sec:motivation}

Students in computer engineering have a growing requirement to have strong knowledge of both embedded systems and computer architectures due to increases in computer engineering careers \cite{cec2016, bls2014}. Unfortunately, many students do not show enthusiasm for these subjects \cite{Ackovska2014,Ackovska2014,Stolikj2011}. 

At \uwo{}, students in all programs in Electrical \& Computer Engineering must take the course ECE 3375 - Microprocessors and Microcomputers \cite{uwo-we-programprogression, eceoutline-ece3375}. For example, shown in \cref{table-course-enthusiasm-ece3375}, for a required course in embedded systems at \uwo{} (ECE 3375) \cite{eceOutlines}, student enthusiasm for the course is extremely mixed. Unfortunately, there does not exist data broken down between computer and software engineering -- both are required to take this course. Additionally, with primitive sentiment analysis applied to \cite{evals:ece3375-2013, evals:ece3375-2014}, shown in \cref{fig:ece-3375-course-sentiment}, we can see that after completion of the course, students are overall happy with the course content in itself as approximately 73\% of students found the experience positive. However, by filtering comments regarding the laboratory sessions and applying the same technique for sentiment as the course, shown in \cref{fig:ece-3375-lab-sentiment}, we find that students are likely very unhappy with the lab components as approximately 33\% stated negative reviews of the component. 

\begin{table}[hb!]
    \centering
    \begin{tabular}{lc|r|r|r}
        \multicolumn{5}{c}{Initial Level of Enthusiasm} \\ \hline\hline
        \multicolumn{2}{l}{Year} & \multicolumn{1}{|c}{2013} & \multicolumn{1}{|c}{2014} & \multicolumn{1}{|c}{Total} \\ \hline
        High          & \%       & 45.6                     & 57.8                     & 53.0                      \\ 
        Medium        & \%       & 54.2                     & 33.3                     & 41.6                      \\
        Low           & \%       & 0.0                      & 8.9                      & 5.4                       \\ 
        Total         & \#       & 59                       & 90                       & 149                       \\ \hline
    \end{tabular}
    \caption{ECE 3375 Course ``Level of enthusiasm before taking the course'' as reported by undergraduate students \cite{evals:ece3375-2013, evals:ece3375-2014}.} 
    \label{table-course-enthusiasm-ece3375}
    \todofig{Update with 2015/2016 data if provided}
\end{table}


\begin{figure}
    \centering
    \begin{subfigure}{.8\linewidth}
        \centering
        \includegraphics[width=.85\linewidth]{img/course-sentiment}
        \caption{Overall Course Sentiment}
        \label{fig:ece-3375-course-sentiment}
    \end{subfigure}
    
    \begin{subfigure}{.8\linewidth}
        \centering
        \includegraphics[width=.85\linewidth]{img/lab-sentiment}
        \caption{Lab Sentiment}
        \label{fig:ece-3375-lab-sentiment}
    \end{subfigure}
    
    \caption{ECE 3375 - Course Sentiment Analysis\cite{evals:ece3375-2013, evals:ece3375-2014}}
\end{figure}

Researchers have found that there are concerns with low-level hardware constructs being taught to high-level Computer Science, Software Engineering and Computer engineering graduates \cite{Ristov2011, Stolikj2011}. Additionally, we can corroborate the following claims regarding student's opinions on microprocessor courses: 
\begin{displayquote}[{\cite[p.~340]{Ristov2011}}]
    \begin{observations}
        \item Disjointed material between lectures, theoretical and practical
        exercises. Students could not transfer knowledge gained from either lectures or theoretical exercises to practical exercises [\dots] 
            \label{quote:ristov:1}
        \item Inappropriate programming and simulation environment. Students faced installation problems [\dots] leading to aversion to presented material and exercises.
            \label{quote:ristov:2}
        \item No ``real world'' application. [\dots] Students [raise] the main question: Why we are learning this, and how and where shall I use it?
            \label{quote:ristov:3}
    \end{observations}
\end{displayquote}
Given our first-hand experiences in both taking and assisting in the instruction of ECE 3375 and ECE 3390, these claims are repeated by students during laboratory exercises and review sessions creating hostility towards the subject matter. In a direct comparison to \cref{quote:ristov:2}, students in ECE 3375 use the WinIDE12Z Integrated Development Environment \cite{winide}, shown in \cref{fig:winide-screenshot-pemicro}. In ECE 3375, students found WinIDE complicated, frustrating and ``outdated'' \cite{evals:ece3375-2013, evals:ece3375-2014} directly supporting both \cref{quote:ristov:2} and \cref{quote:ristov:3}. Additionally students found it difficult to move from lecture material that presented embedded systems as ``translucent'' devices composed of inner components to the literal ``black box'' of the physical hardware shown in \cref{fig:hc12-board}. 

\begin{figure}[!ht]
    \centering
    \includegraphics[width=0.7\linewidth]{img/winide-screenshot-pemicro}
    \caption{WinIDE12Z Integrated Development Environment \cite{winide-screenshot}}
    \label{fig:winide-screenshot-pemicro}
\end{figure}

\begin{figure}[!hb]
    \centering
    \includegraphics[width=0.8\linewidth]{img/board-on-cropped}
    \caption{Freescale S12CPUV2 development board built by Western Engineering Electronics Shop for use in ECE 3375}
    \label{fig:hc12-board}
\end{figure}

With sentiment analysis and the comments provided by students, the author extracted several propositions for improving laboratory sessions: 

\begin{propositions}%[label={\textbf{\theproposition \arabic*}}]
    \item Increase the number of laboratory sessions 
        \label{prop:add-lab-sessions}
    \item Add tutorial sessions to course schedule
        \label{prop:add-tutorial-sessions}
    \item Change course material to use modern processors
        \label{prop:change-course-materials}
    \item Replace laboratory microcontrollers with different hardware
        \label{prop:replace-hardware} % REPEAL AND REPLACE!11!!11
    \item Improve and/or replace software within laboratory
        \label{prop:replace-software}
\end{propositions}

All of these considerations were bought to the course instructor and the following limitations were noted for each. For \cref{prop:add-lab-sessions,,prop:add-tutorial-sessions}, while the addition of more hands-on experience has been shown to improve student outcomes \cite{Ristov2011, Stolikj2011}, due to the large amount of financial resources required to add more laboratory sessions, this option is likely unavailable for many programs. In addition, due to the layout of the current course laboratory sessions, there does not exist enough weeks in the 12-week course to add more sessions given current staffing resources. It is not absurd to assume that these issues are prevalent at other institutions. Thus, both \cref{prop:add-lab-sessions,,prop:add-tutorial-sessions} can not be implemented given financial and temporal restrictions.

\Cref{prop:change-course-materials} asks the instructor to change the current course material to use more modern hardware and software architecture. There are several reasons why this is not beneficial to the improvement of learning outcomes. The current architecture used is the \hcmodel{}. This architecture provides a simple CISC ISA with only \numberstringnum{4} 16-bit addressable registers (\verb|SP|, \verb|PC|, \verb|X|, \verb|Y|); \numberstringnum{2} \numberstringnum{8}-bit accumulators (\verb|A|, \verb|B|) and multiple addressing modes \cite{hc12Manual2006}. The software abstractions of the \hcmodel{} assembly is similar to current Intel\textregistered{} x86 processors \cite{intel2017}. The parallelism between architectures allows instructors to use more simple architectures that still allow students to apply concepts to newer architectures while simultaneously reducing the cognitive ``base knowledge'' required to succeed. Using simpler architectures like the Ultimate Reduced Instruction Set Computer (URISC, \cite{Mavaddat1988}) has been shown to be very beneficial to improving student outcomes \cite{Nakamura2013, McLoughlin2010, Mavaddat1988, Djordjevic2005, Garcia2009}. Additionally, by changing course materials, it leads into changing the laboratory hardware which \cref{prop:replace-hardware} discusses. Unfortunately, replacing laboratory equipment requires significant investment and with justifications for \cref{prop:change-course-materials} showcasing that there is not noticeable improvements in learning outcomes, the cost-benefit analysis does not prove worthwhile to adjust both \cref{prop:change-course-materials,,prop:replace-hardware}.

The last proposition, \cref{prop:replace-software} is the most attainable. Researchers have found that improving software technology in the classroom and laboratory can vastly improve the learning outcomes of students while improving engagement \cite{Ackovska2014, Stolikj2011, Ristov2011, Ristov2014, Nikolic2009, Skillen2011, Tappan2009, Djordjevic2005, cec2016}. Students within ECE3375 are immediately presented with the interfaces shown in \cref{fig:winide-screenshot-pemicro}. Students have expressed immense frustrations with the software when used in the laboratory exercises and often encounter hardware issues that are not caused by their own work \cite{evals:ece3375-2013, evals:ece3375-2014}. Other researchers have found hardware solutions problematic with hardware-specific problems common such as broken peripherals and unconnected pins \cite{Ackovska2014}. These errors are hard to debug, if not impossible for students with no prior knowledge. Teaching assistants and instructors are often unable to debug these issues without the aid of technicians. These hard-to-diagnose hardware issues lead to strained teaching resources and frustrations, supporting a large potential gain by replacing this ageing software with new tooling. Given these justifications, this thesis looks to investigate the requirements surrounding replacing the current software in use within the ECE 3375 course and propose a configurable solution for others to utilize in other courses. 

\section{Problem Statement}
\label{sec:problem-statement}

Our analysis of the current state of the ECE 3375 course showcased the pervasive dislike and discomfort with the current laboratory exercises \cite{evals:ece3375-2013, evals:ece3375-2014}. Given previous analysis, the author chose to improve the current software used within the laboratory exercises and the classroom. To do this, a current survey of known technologies must be done to ascertain whether a new solution should be created or an existing solution will provide the requirements. Therein the formal problem becomes: 
\begin{displayquote}
    Can a solution be provided to students that is configurable for hardware and software interfaces and showcases the connection between hardware and software while focusing on pedagogy rather than industry performance? 
\end{displayquote}
While the formalization is accurate, \textquote[\cite{Skillen2011}]{striking the right balance between teaching sufficient details of hardware components and their working principles, and important theoretical concepts useful for programming the computer is always a challenge.} From reviewing many existing software solutions to this problem, we concluded that in order to be successful a solution must meet the following requirements:

\begin{requirements}
    \item\label{req:personal} Must be available for use outside of the laboratory on personal computers
        
    \item\label{req:configuration} Provide a student configurable system for different ISAs including configuration of:
        \begin{requirements}
            \item microcode and assembly instructions
            \item execution semantics (i.e. how a processor executes code)
            \item internal hardware components and connections
            \item external peripherals
        \end{requirements}
        
    \item\label{req:pedagogical} Focus on pedagogy over simulation accuracy
       
    \item\label{req:simulations} Simulations must allow: 
        \begin{requirements}
            \item viewing of the current state of hardware components
            \item stepping at the assembly and microcode level
            \item setting of breakpoints to ease debugging
            \item connecting ``external'' peripheral components
        \end{requirements}
    
    \item \label{req:modern} Provide a modern user interface that is similar to current high-level programming IDEs
        
\end{requirements}

\Cref{req:personal} outlines that the software must work on personal computers. StackOverflow's Developer survey \cite{StackOverflowSurvey2016} found that approximately 47.9\% of those surveyed used a non-Microsoft Windows environment (macOS or Linux). The metrics recorded for 2016 desktop operating systems are shown in \cref{fig:stack-overflow-os-share}. While other metrics show that over 80\% of computers are Microsoft Windows \cite{StatCounter2017}, the author believes the StackOverflow survey is better given students in computer and software engineering programs likely go on to become hardware or software developers \cite[Sec.~II.~Developer~Profiles]{StackOverflowSurvey2016}. Further, StackOverflow has found the year-over-year change to show a decrease in Microsoft Windows-based operating systems between 2013 and 2016 \cite{StackOverflowSurvey2016}. StackOverflow's results imply that this project's solution must run on all three major operating systems with minimal effort to allow students to perform out-of-classroom exercises. 

\newcommand{\ostitlecell}[1]{\parbox[t]{2mm}{\multirow{6}{*}{\rotatebox[origin=c]{90}{\textbf{#1}}}}}

\begin{figure}[!hp]
    \centering
    \includegraphics[width=.7\linewidth]{img/stack-overflow-os-share}
    
    \begin{table}[H]
        \centering
        \begin{tabular}{llr|llr|llr}
            \multicolumn{9}{c}{\thead{Operating System}} \\ \hline
            \ostitlecell{Linux} & Ubuntu & 12.3   & \ostitlecell{Apple} & macOS & 26.2   & \ostitlecell{Windows} & 7    & 22.5  \\
            & Debian & 1.9  & &     &     & & 10      & 20.8  \\
            & Fedora & 1.4  & &     &     & & 8       & 8.4   \\
            & Mint   & 1.7  & &     &     & & Vista   & 0.1   \\
            & Other  & 4.4  & &     &     & & XP      & 0.4   \\
            & \textbf{Total} & \textbf{21.7} &  & \textbf{Total} & \textbf{26.2} & & \textbf{Total} & \textbf{52.1} \\ \hline
        \end{tabular}
    \end{table}    
    \caption{StackOverflow 2016 developer survey results for desktop operating systems \cite[Sec.~VIII.~Desktop~Operating~System]{StackOverflowSurvey2016}}
    \label{fig:stack-overflow-os-share}
\end{figure}

\let\ostitlecell\@undefined

As stated within the Computer Engineering Curricula 2016:
\begin{displayquote}[{\cite[p.~32]{cec2016}}]
    One area of concern to the computer engineer is the software/hardware interface, where difficult trade-off decisions often provide engineering challenges. Considerations on this interface or boundary lead to an    appreciation of and insights into computer architecture and the importance of a computer’s machine code. At this boundary, difficult decisions regarding hardware/software trade-offs can occur, and they lead naturally to the design of special-purpose computers and systems.
\end{displayquote}
\noindent \Cref{req:configuration} enables students to design and configure all of the hardware/software component interactions. This enables students to have a stronger understanding of ISA design. While many tools have this flexibility (discussed later in \cref{sec:review}), many of the currently used, industry-grade solutions provide over-complicated and powerful options that are overwhelming for students. It is expected that disorientation by complication of user interfaces for users may have an impact on performance, production, motivation and morale of users \cite{Chalmers2003}. This ``user hostility'' must be reduced for students as they are less likely to have the motivation that an industry professional may have \cite{Djordjevic2005}. As the focus of this thesis is to improve student learning outcomes, the amount of options are not relevant for learning must be reduced as a trade-off in favour of pedagogical improvement instead of accuracy as listed in \Cref{req:pedagogical}. 

\Cref{req:simulations} outlines the software requirements that simulations must meet while still respecting \cref{req:pedagogical}. Simulations are an important practical tool for students to see the inner workings of hardware components and attempts to increase the transparency into the long-standing ``black-box'' of hardware for embedded systems courses. For computer architecture courses, simulation tools are required for debugging and implementing ISAs. 

Lastly, over-arching \Cref{req:modern} is to attempt to reduce the overhead gap of working with new tools by reducing differences in features for students. Reducing the overall cognitive load by integrating similar styles and controls to existing tools within interfaces will reduce the cognitive load for students which should in turn improve learning outcomes \cite{Chalmers2003, Mavaddat1988}.

\section{Contributions}

This thesis is presented in an additional \numberstringnum{\numexpr\totvalue{chapter} - 1\relax} chapters. This thesis discusses technical topics regarding: existing and future pedagogical simulation technologies; development of web applications on modern platforms; building cross-platform native applications; and integration of scripting languages as configuration engines within a native application. We summarize the contributions to each in the following sections.

\paragraph{Survey of other and previous works.} Within this section, we provide a quantitative survey of existing simulator technologies in use today analysed against the requirements outlined in \cref{sec:problem-statement}.

\paragraph{procsim.scala: a Scala-based event-driven processor simulator for the modern web.} We investigate utilizing Scala and it's compiler back-end Scala.js to build a massively parallel processor simulator for the modern web. Within this investigation, we showcase the following contributions: 
\begin{itemize}
    \item A design for an asynchronous actor-based model for simulation of a custom processor architecture using distributed computing framework Akka
    \item A runtime, compiled VHDL-like DSL for specification of instructions within a custom ISA on top of Scala in the Java Virtual Machine and Scala.js
\end{itemize}

\paragraph{Developing cross-platform C++ applications.} We discuss an anecdotal account of refactoring the hc12sim project into a modern C++ environment. We discuss topics involving: developing a multi-platform project and providing adequate quality assurance, improving CMake through custom build scripts, and improving application build times through optimization of build artefact selection. The contributions outlined within this section are: 
\begin{itemize}
    \item An automated platform-specific testing infrastructure using Virtual Machines provisioned with Vagrant and Oracle VirtualBox automated with Jenkins' Pipeline architecture
    \item A collection of project-based CMake scripts to isolate build targets while improving build times, reducing manual configuration and modularizing test specification
\end{itemize}

\paragraph{Lua-based configuration-driven processor simulation.} We discuss the use of scripting engines in a native C++ application and their integration points; utilizing Lua as a scripting language within an application; implementation of an application while developing integration tools in parallel; and the pedagogical gains of a high-level language utilized for teaching low-level concepts. We provide the following contributions:
\begin{itemize}
    \item A design for a runtime configuration specification for custom processor architectures through Lua scripts
    \item A state-machine-like representation of a processor instruction execution specified through ``compiling'' a Lua function to microinstruction events
\end{itemize}
%
%\section{Outline} 
%
%\todo[inline]{Is the previous section better and more direct?}
%
%In \cref{ch:scala-akka} we discuss a the development of a Scala-based processor running on top of the Akka distributed actor framework within a web browser environment through Scala.js. In \cref{ch:cross-platform}, we discuss the steps taken to develop a cross-platform C++ application through continuous integration tools and consistent Virtual Machine specification through Vagrant. Additionally, \cref{ch:cross-platform} discusses a custom set of CMake scripts to automate specification of project components within a multi-project build environment that improves isolation of targets  and flexible build mechanisms. In \cref{ch:lua}, we develop a C++-based simulator built from the original hc12sim project covered in \cref{sec:review-prev-hc12}. We define a configuration specification based on Lua, an embedded scripting language, and a mechanism for specification of instruction execution that utilizes Lua's scripting engine. Lastly, \cref{ch:conclusion} provides conclusions and recommendations for future work. 

\chapter{Previous and Survey of Other Works}
\label{ch:prev-other-work}

\section{Previous Work: hc12sim}

In 2013, the author worked collaboratively with colleague and fellow student Ramesh Raj to build a behaviourally accurate simulator for the M68HC12 processor\footnote{The author's capstone work is outlined in \cite{Brightwell2013}}. This work was completed as part of the capstone project for the author's Computer Engineering degree from \uwo{}. This software was worked on over the course of a year and had the following features: 

\begin{enumerate}
    \item Included all basic memory elements (e.g. registers, memory layout)
    \item Compile-time configurable microinstructions (addressing modes were not configurable)
    \item Real-time simulation speeds for the \hcmodel{}
    \item Debugging of assembly code with runtime viewing of internal hardware components
    \item Accurate CISC addressing modes 
    \item Full assembler creating exact machine code
    \item Simple IDE for writing M68HC12 assembly code, compiling and simulating it (shown in \cref{fig:hc12ide-invalid-instruction})
\end{enumerate}

\begin{figure}[!ht]
    \centering
    \includegraphics[width=0.6\linewidth]{img/hc12ide-invalid-instruction}
    \caption{hc12ide Integrated Development Environment showing a student error.}
    \label{fig:hc12ide-invalid-instruction}
\end{figure}

\subsection{Software Design}

This project was written in C++ utilizing modern C++11 features such as lambdas, closures and cross-platform libraries (e.g. Qt and Boost\footnote{Available: Qt (\url{https://www.qt.io}) and Boost (\url{https://boost.org/})}). During this work, the primary design goals were: 

\begin{enumerate}
    \item Support all three major platforms: Microsoft Windows, Mac OS X, and GNU/Linux
    \item Separate compiler and simulation components
    \item Completely decouple any IDE from simulation
    \item Heavily unit-test all software components for validity
\end{enumerate}
\todo{Compare these goals to reqs from \cref{sec:problem-statement}}

These lead into further requirements for any technology used (in order of priority): 

\begin{enumerate}
    \item User interface library available
    \item Access to low-level types
    \item Fast enough for simulation
\end{enumerate}

While initially writing the software in Java utilizing Java's Swing Toolkit and the Java Standard Library, it was rewritten in C++ due to access of low-level unsigned types, method references and multithreading capabilities.

\subsubsection*{Instructions}
\label{hc12sim:instruction-generation}

Due to the M68HC12 having over 200 instructions with multiple addressing modes for each\cite{hc12Manual2006}, the simulation and assembler code put heavy emphasis on generating code to simplify writing and adding instructions. This allowed for generating multiple classes that had all of the meta-information required for all stages of a program (writing, assembly, programming, execution). In order to handle all of the generation required, a custom user interface was created to easily load and store data and generate C++ code. An image of the tool at this time is shown in \cref{fig:hc12sim-instruction-generater-ui}. As shown, the execution code is written in C++ and written beside the encoding of different addressing modes\footnote{See \cite[Table~3-1,~p.~30]{hc12Manual2006} for a table outlining the addressing modes available for the M68HC12}\todo{Should the full table be included from \cite{hc12Manual2006}?}. In addition to the user interface, the generator back-end was built into a command-line tool that allowed the generation to be run at compile time when hooked into build scripts (in this case, CMake\footnote{CMake: \url{https://cmake.org}}). The use of code generation became extremely prevalent in future work. 

\begin{figure}[h!t]
    \centering
    \includegraphics[width=0.6\linewidth]{img/hc12sim-instruction-generater-ui}
    \caption{hc12sim Instruction Generator showing the meta information for addressing modes, execution, and output.}
    \label{fig:hc12sim-instruction-generater-ui}
\end{figure}

\subsubsection*{Hardware Simulation}
\label{hc12sim:hardware-simulation}

Physical ``components'' of the simulation were all written as custom classes. For example, the \verb|A| and \verb|B| registers were of \cxxinline{class Register<size_t width>} where the \verb|width| template parameter defined the bit width of the \cxxinline{Register}. Unfortunately, any special behaviours were ``decorated'' on top of the base type creating many different types depending on how the underlying component was used. This created confusing class hierarchies and directly contradicts the advice of the Decorator pattern\cite[p.~175]{go4}. Of particular interest was the design of the \cxxinline{Timer} and \cxxinline{TimerDependant} types which allowed for completely concurrent, event-based timing. An example interaction is shown in \cref{fig:hc12sim-timer-seq}. While these concurrent timing components are fast and well designed, it creates problems with discrete execution as each operates concurrently in separate thread contexts and can not be ``stepped'' through without executing all of the threaded operations simultaneously. In addition, each \cxxinline{TimerDependant} adds a thread of execution adding runtime overhead to simulation while not adding any functionality to the simulation.

\begin{figure}[!hp]
    \begin{minipage}{.5\linewidth}
        \centering
        \includegraphics[width=\textwidth]{img/hc12sim-timer-sequence} 
        \todofig{Replace with visio diagram}
    \end{minipage}%
    \begin{minipage}{.5\linewidth}
        \begin{enumerate}
            \item A \cxxinline{Timer} object is instantiated and started, it will enter sleep state
            \item Two \cxxinline{TimerDependant} objects are created with execution methods bound within.
            \item The \cxxinline{Timer} object binds the two \cxxinline{TimerDependant} objects to itself
            \item In the \cxxinline{Timer}'s thread, it wakes and asks the Operating system to simultaneously signals all \cxxinline{TimerDependant} objects to wake
            \item Synchronously, all the \cxxinline{TimerDependant} objects:
            \begin{enumerate}
                \item Run it's own execution method within it's thread
                \item Return to a sleep-state and wait for next notification
            \end{enumerate}
        \end{enumerate}
    \end{minipage}
    \caption{Sequence of events for \cxxinline{Timer} and \cxxinline{TimerDependant} interaction\cite{Brightwell2013}} 
    \label{fig:hc12sim-timer-seq}
\end{figure}

\subsubsection*{Execution}

The simulator implemented execution through an \cxxinline{class Executor} that asynchronously executes the execution scheme specified by the M68HC12\cite[Sec.~4,~p.~47]{hc12Manual2006} and implemented in terms of \cite[p.~59]{Vahid2002}. The \cxxinline{Executor} utilized a fetch, decode, fetch (if needed), execute loop for executing all instructions. The actual execution of an instruction was completed by code written in the instruction generation step (see \cref{hc12sim:instruction-generation} for more information). However, the generated code is run within a separate thread context to keep the ``clock'' of the system running at a consistent speed. Of particular note, this simulation could not handle interrupts or pipe-lining. The full sequence of actions of this execution unit is shown in \cref{fig:hc12sim-execunit-sequence}.

\begin{figure}[!ph]
    \centering
    \includegraphics[width=.9\linewidth]{img/hc12sim-execunit-sequence}
    \caption{Sequence diagram of Execution process within Simulator}
    \label{fig:hc12sim-execunit-sequence}
    \todofig{Replace with visio diagram}
\end{figure} 

\subsubsection*{Project Iteration}

While the hc12sim project was a success by most metrics, it missed major features such as interrupts and pipe-lining. It is tailored very specifically to the Freescale M68HC12 processor. Lastly, it does not support attaching peripheral devices (e.g. motors, LEDs). 


\section{Evaluation of current simulation technologies} 
\label{sec:simulator-survey}

\todo{Better intro to this section}Presented in the following sections are a series of similar works to the space 

\subsection{ShelbySim}

\cite{Tappan2009}

ShelbySim is an education-oriented software system for designing, simulating, and evaluating computer-engineering based applications. ShelbySim was designed surrounding a new Java-like programming language including a compiler explicitly built around providing extensive diagnostic information such as logging, tracing, and inspection capabilities. These tools provide students with raw data for quantitative analysis, evaluation and reporting of their designs. 

The software is open-source, though not available, and is written using Java allowing full operating system independent support. Additionally, 3D visualized results are provided for viewing developed components. ShelbySim is broken down into three subcomponents:
\begin{enumerate}
    \item Software component - a custom programming language (Shelby), a compiler, and an interpretation runtime
    \item Hardware component - filling a similar niche to MultiSim, but with tight integration with Shelby and its underlying tracing. Additional support exists for external component integration
    \item Simulation component - providing a deterministic and stochastic approach for inputs into custom hardware versions
\end{enumerate} 

Provides evaluation criterion for students components and underlying systems. The simulated components have parameters that are modifiable through switches and sliders (e.g. \{on, off\} or a range from 0 - 100\%). This gives students metrics to evaluate their designs. Additionally, outputs are exported at runtime to Comma Separated Value (CSV) files allowing for more in-depth analysis with external programs such as Microsoft Excel or MATLAB. This gives a flexible and realistic testing environment for student learning. 

\subsubsection{Talking Points}

\begin{itemize}
    \item Most software does not focus on pedagogy
    \item Industry software hides underlying information (rightly so for them)
    \item Visualization gives "less opaque" view of the system
    \item Language 
    \item Focuses on compiler semantics, low-level detail implementations (e.g. motor characteristics)
\end{itemize}

\subsection{CPUSim}

\cite{Skrien2001}

CPU Sim is a Java CPU simulator written for use within a classroom environment. CPUSim allows students to design, modify, and compare various computer architectures at the register-transfer level and higher. Additionally, students may write and debug assembly code for custom architectures. The JavaFX front-end allows for viewing CPU internal components (e.g. RAMs and Registers) while stepping through programs. CPUSim allows students to encode and decode values through a user interface for machine codes. 

CPUSim also allows students to specify microinstructions that are combined to create full assembly-level instructions. This forces students to describe and think about what actions they need for their ``higher-level'' instructions. 

CPUSim's design is flexible given the Java-based system allowing students to work on multiple different platforms easily. Additionally, it is written such that many aspect of the software may be customized. Unfortunately, due to the decision to utilize the Java Virtual Machine, interfaces with lower-level components such as serial ports is difficult. 

\subsubsection{Talking Points}

\begin{itemize}
    \item Java-based
    \item Full-debugger
    \item Specify microinstructions (microcodes) instead of assuming them
    \item Code is older, tightly coupled and problematic
    \item 
\end{itemize}

\subsection{EASE}

\cite{Skillen2011}

\blockquote[Skillen2011]{Striking the right balance between teaching sufficient de-tails of hardware components and their working principles, and important theoretical concepts useful for programming the computer is always a challenge.}

Found most of the simulators in \cite{Nikolic2009} were inadequate for teaching simulation architecture courses, driving the work in EASE. Many were used for circuitry/RTL level work thus not good enough for project at hand. 

The following characteristics are most important: 

\begin{enumerate}
    \item Support of more than one architecture to illustrate CISC, RISC and URISC
    \item Designed in a modular way to allow for extension of the simulation (adding new instructions)
    \item Source code available
    \item Portable
\end{enumerate}

All of which are now (or will be) fixed in future versions of CPUSim. The article compared approximately version 3.1 (\cite{Skrien2001}). 

Requires the following further improvements: 

\begin{itemize}
    \item Add built-in editor
    \item No provided documentation or samples
    \item 
\end{itemize}

\subsubsection{EASE vs. CPUSim}

Architectures are specified outside the simulation itself as a Java Library -- disconnected from the simulator/software itself. 

Architectures are bound to specific models that are specified by EASE, (e.g. CISC (similar to x86), RISC and a URISC). These are tightly coupled to EASE itself. 

Based on descriptions of the internal mechanics, it is tightly coupled to the current structure and not flexible in changing internals -- though, the source is not available. 


\subsubsection{Talking Points}

\begin{itemize}
    \item Java-based
    \item CPUSim is directly compared in this paper
    \item No source available (yet claims it is..)
    \item Dead-ware
    \item Tightly coupled to specified architectures
    \item Focuses on assembly in these architectures
    \item Ships with URISC architecture (subleq)
\end{itemize}

\subsection{Nikolic}

\cite{Nikolic2009}

Mixture of discussion surrounding architectures and organization -- less on simulation of actual assembly-level programming.  


\subsubsection{Talking Points}

\begin{itemize}
    \item Java-based
    \item CPUSim is directly compared in this paper
    \item No source available (yet claims it is..)
    \item Dead-ware
    \item Tightly coupled to specified architectures
    \item Focuses on assembly in these architectures
    \item Ships with URISC architecture (subleq)
\end{itemize}

\subsection{A Perspective on the Experiential Learning of Computer Architecture}

\cite{McLoughlin2010, Nakamura2013}

Discussion of an MSc. curriculum based around designing a CPU called ``TinyCPU'' \cite{McLoughlin2010} and an extension ``TinyCSE'' \cite{Nakamura2013}. The MSc. program detailed is very similar to \uwo's computer engineering undergraduate program in content/topics. 

\subsubsection{Design considerations}

Utilizes a simple dual-BUS architecture for address and data. Uses memory-mapped ``I/O space'' controller for a port-mapped I/O scheme. An interrupt controller is built consisting of a single register, \verb|intr| that stores information about what device interrupted. The authors used a simplified interrupt model in which only one interrupt is supported at a given time. The addition of the interrupt support required the addition of ``RETURN'' and ``CALL'' instructions to support subroutines. Any interrupt implementation must have these machine instructions specified to give the controller the ability to change execution flow whilst maintaining state of the machine. 


\subsubsection{Talking Points}

\begin{itemize}
    \item Verilog HDL-based
    \item Requires use of complex suites like Modelsim or Altera
    \item To be used directly on FPGAs
    \item Comes with a C compiler/assembler
    \item Proved implementation of multiple device controllers
    \item Bound to a single architecture
\end{itemize}

\chapter{procsim.scala: a Scala-based processor simulator for the modern web}
\label{ch:scala-akka}

\newcommand{\scalainline}[1]{\mintinline{scala}{#1}}
\newcommand{\akkaActor}{\scalainline{Actor}}

\section{Introduction to Scala and Akka}

procsim.scala attempted to port work of hc12sim to Scala and Akka while providing incremental feature improvements surfaced by reviewing other projects in \cref{sec:review}\footnote{See \cref{sec:review-prev-hc12} for an extended discussion on the original hc12sim}. In recent years, Scala has emerged as a powerful language used in big data and scalable web applications thanks to Apache Spark\footnote{Apache Spark is available at: \url{https://spark.apache.org/}} and Lightbend Akka\footnote{Lightbend was formerly known as Typesafe. Akka is available at: \url{https://akka.io}} \cite{Karau2015, Alexandrov2014, Singh2015}. Scala is a multi-paradigm programming language supporting functional and object-oriented style programming on both the Java Virtual Machine (JVM) and within a JavaScript environment such as a web browser \cite{Scala-Lang}. On the JVM, Scala can interoperate with Java programs without any code changes and in a JavaScript environment most Java functionality is available \cite{Scala-js2015,Doeraene2017}. One of Scala's original use-cases was as a less verbose and more functional version of Java that runs within the same Java infrastructure. Thanks to Scala's implicit parameters \cite{Scala-ImplicitParameters}, custom operator overloading \cite{Scala-Operators} and optional post-fix notation \cite{Scala-MethodInvocations} developers can create Domain-Specific Languages within Scala with ease. For example, \cref{lst:scala-dsl-basic} show cases a DSL that was written to emulate BASIC (a 50 year-old language) showcasing how flexible Scala's syntax can be. In regards to processor simulation, all of the simulators surveyed provide at least one mechanism for loading and saving machine state to a persistent storage (e.g. XML files for \cite{Skrien2001, Black2013}). A DSL-based configuration allows specification of a system without extraneous syntax designed specifically for this purpose. Utilizing a DSL could allow students to specify their machines programmatically in addition to a graphical user interface improving the amount of configuration, reuse and flexibility available. This provides an excellent approach to improving \cref{req:configuration}'s configuration requirements.

\begin{listing}[b!]
\begin{minted}{scala}
object SquareRoot extends Baysick {
    def main(args:Array[String]) = {
        10 PRINT "Enter a number"
        20 INPUT 'n
        30 PRINT "Square root of " % "'n is " % SQRT('n)
        40 END
        
        RUN
    }
}
\end{minted}
\caption{An example DSL source written in Scala to emulate BASIC called ``Baysick'' \cite{FogusBaysick}}
\label{lst:scala-dsl-basic}
\end{listing}

Akka is a Scala framework that provides an implementation of the Actor model \cite{Agha1985} that was originally written by Haller and Odersky in \cite{Haller2009} for the Scala Library. \cite{Haller2009}'s original implementation makes use of an event and message-driven programming model to lower thread counts and alleviate lock contentions compared to traditional threading models. \Cref{fig:doyle2014-akka-actor-model} shows a simple diagram of actors sending messages bi-directionally between each other. Each actor has a conceptual ``mailbox'' that they can store and process messages from. The initial actor implementation by \cite{Haller2009} was expanded to become Akka which is now sponsored by Lightbend. Akka's major revisions to \cite{Haller2009} includes additional abstractions to remove locational knowledge such that actors ``live'' wherever they are best suited. Additionally, Akka's scheduling mechanisms control \akkaActor{} execution patterns in an idealistic and a deterministic way \cite{TypesafeAkka2015}. For example, two \akkaActor{} instances can communicate with each other regardless of their physical location on the same node of a distributed system or completely geographically separated by a network -- an individual actor neither knows nor cares where a message comes from or is sent to. Akka provides fault tolerance through supervisor semantics utilizing the ``let-it-crash'' paradigm for faults popularized by the Erlang programming language in telecommunication systems since 1985 \cite{Armstrong2010}. These features on top of the actor model create a very scalable technology that maps simply to microservice-driven architectures. While Akka's main goal is to create microservice architectures and scalable applications, we intended to utilize this model to instead implement an event-driven simulation model.

\begin{figure}[bh!]
    \centering
    \includegraphics[width=0.7\linewidth]{img/doyle2014-akka-actor-model}
    \caption{Graphical representation of three Actors sending bi-directional messages \cite{DoyleAkka2014}.}
    \label{fig:doyle2014-akka-actor-model}
\end{figure} 

\section{Event and Message-driven Simulation}

Several projects surveyed utilized event-driven models of simulation \cite{Nakamura2013, McLoughlin2010, Garcia2009}. As discussed in \cref{sec:review-summary}, event-driven models provide a simplification to maintenance of state within an application. In a traditional object-oriented application, each object maintains its own state performing create, read, update and delete operations on itself and acts on other objects to trigger their state changes. By contrast in a traditional event-driven model, the ``owner'' of state still remains with components, but program flow is dictated by utilization of an event loop that processes events and asks components to ``react'' to these events further creating more events. This model is traditionally found in graphical user interface applications as it simplifies flow of a user-driven system. Lightbend coined the phrase ``reactive'' in regards to software by creating a manifesto known as the ``Reactive Manifesto'' that states the qualities of a ``reactive'' application \cite{ReactiveManifesto2014}. The ``Reactive Manifesto'' details message-driven applications as: 
\begin{displaycquote}{ReactiveManifesto2014}
    Reactive [systems] rely on asynchronous message-passing to establish a boundary between components that ensures loose coupling, isolation and location transparency. This boundary also provides the means to delegate failures as messages. [...] Location transparent messaging as a means of communication makes it possible for the management of failure to work with the same constructs and semantics across a cluster or within a single host. Non-blocking communication allows recipients to only consume resources while active, leading to less system overhead.
\end{displaycquote}
While the majority of this definition deals with distributed systems, the same benefits are gained with local applications. A processor simulator does not require elasticity or load management; it does however require the ability to process messages quickly and in such a way that modules do not block each other as electronics all operate in parallel. These ideals were transcribed into a model of a processor in which all components interact through ``signal'' messages between each other rather than direct interactions. This opens up the opportunity for much more accurate simulations while simultaneously reducing the effort to interact between modules. These relationships allow simulations to have higher fidelity and ease the time required to implement features lending well to both \cref{req:simulations,,req:pedagogical}. 

\section{Scala.js - Scala transpiler for JavaScript}

Scala.js is a ``compiler back-end'' for the Scala compiler that transpiles Scala code to JavaScript so that it can execute within a web browser. At the time of creating the procsim.scala project, the most recent release of the Scala.js transpiler was version 0.6.5 \cite{Scala-js2015}. Version 0.6.5 allowed for most Scala applications to be cross-compiled to the Java Virtual Machine and JavaScript applications with little-to-no code changes. Through Scala.js, a simulation written in Scala could be run easily on both a desktop and web platform opening up the opportunity to utilize Scala for both a front-end client and a web service. \Cref{req:personal,,req:modern} support the use of a web-based solution due to the prevalence of web browsers compatibility in a cross-platform environment and current student's relative comfort with web-based applications. Given our's strong Java and Scala background, utilizing Scala.js to build a fully-featured web application on top of the Scala and Akka platforms was an excellent candidate for this thesis project.

\section{Implementation of procsim.scala}

\subsection{VHDL-like DSL configuration and runtime-based instructions}
\label{sec:sec:procsim-scala:configuration}

The first goal of procsim.scala was to introduce a DSL for defining modules and instructions within Scala. By creating a DSL that mimics VHSIC Hardware Descriptor Language (VHDL), a commonly used HDL language at \uwo{}, for students to define instructions within their processors, the intention was to have all modules pre-built and configurable but instructions are specified by students to give a transparent look into microcoding for a processor architecture. Given the compile-time configuration presented by the original hc12sim project in \cref{sec:hc12sim:instruction-generation}, it was our intention to allow students to have the same level of reconfigurability at runtime instead. The largest benefit to runtime over compile-time configuration is removal of the requirement to install a large compiler tool-chain and wait for the rest of a large application to compile -- in the order of 10 to 40 minutes depending on hardware utilized\footnote{On an Intel\textregistered{} i7-3770k, a clean build of the hc12sim project takes approximately 10 minutes, incremental builds were faster depending on the files modified.}. 

The author achieved runtime-configuration of a system by utilizing Scala's built-in reflection toolbox \cite{Scala-Reflection}. Scala defines reflection as \textquote[\cite{Scala-Reflection}]{the ability of a program to inspect, and possibly even modify itself.} Self modification of running code is the primary feature required for runtime configuration of a model without utilizing separate compiler tool-chains and processes. Scala's reflection tools allow for a developer to write code as a \mintinline{scala}{String} and execute the result, storing any and all state produced by the ``sub-program.'' This feature allows students to specify code in a configuration editor interface and have the software provide Scala-based compilation feedback and JVM-level performance. Two listings are provided: \cref{lst:procsim-scala:concrete-instruction-def,,lst:procsim-scala:reify-instruction-def}, each shows the definition of the same simple instruction. The syntax used for assignment is meant to mimic the syntax used in VHDL to ``connect'' two modules. \Cref{lst:procsim-scala:concrete-instruction-def} utilizes a normal Scala class to define these operations, but \cref{lst:procsim-scala:reify-instruction-def} uses runtime-reification to parse a \scalainline{String}, producing a result with the same functionality as \cref{lst:procsim-scala:concrete-instruction-def}. In small micro-benchmarks, an initial overhead for parsing the \scalainline{String} and compiling it to run within the JVM at runtime was found, however there appears to be no performance penalty to execute code found within a reified construct. These test benches utilized primitive timing mechanisms and accounted for the JVM's JIT warm-up time. Micro-benchmarks proved that it is possible to have complete runtime-based, and compiled configurations within a Scala application without utilizing a secondary compiler tool-chain. The examples shown have the ability to be expanded to mimic the capabilities found in simulators like TinyCSE \cite{Nakamura2013, McLoughlin2010}, reducing the overall knowledge required but still giving students the feeling of ``hands-on'' instruction modification meeting configuration \cref{req:configuration}.

\begin{listing}[ht!]
    \todo[inline]{Fix sub-listing and references}
\begin{minipage}{0.45\textwidth}
    \begin{minted}{scala}
    import net.navatwo._
    
    class Concrete extends Instruction {
    def exec(cpu: CPU) = {
    import cpu._
    A <= 5     // set A = 5
    B <= A + 1 // set B = A + 1 (6)
    }
    }
    val cIns = new Concrete
    \end{minted}
    \caption{Simple instruction that initializes values in registers defined through a Scala class definition.}
    \label{lst:procsim-scala:concrete-instruction-def}
\end{minipage}
\begin{minipage}{0.45\textwidth}
    \begin{minted}{scala}
    import scala.reflect.runtime.{universe => u}
    val tb = mirror.mkToolBox()
    val tree = tb.parse("import net.navatwo._; " +
    "new Instruction { " +
    "def exec(cpu: CPU) = {" +
    "import cpu._;" +
    "A <= 5; " +
    "B <= A + 1;" +
    "}" +
    "}; ")
    val sIns = tb.eval(tree).asInstanceOf[Instruction]
    \end{minted}
    \caption{String-based reified instruction that mimics \cref{lst:procsim-scala:concrete-instruction-def}'s functionality.}
    \label{lst:procsim-scala:reify-instruction-def}
\end{minipage}
\end{listing}

\subsection{Akka-based Components}

The second challenge was utilizing the Akka framework to produce hardware modules that based on Akka's \akkaActor{} model. Within the original description of the Actor model, Agha states:
\begin{displaycquote}{Agha1985}
    A processor is a physical machine while a process is an abstract computation. From operating systems, we know that we may improve over-all performance of a processor by executing several processes concurrently instead of sequentially.
\end{displaycquote}
This quote is both ironic and informative as when considering hardware simulation, each module should be developed as an individual asynchronous entity. Each has input and output connections and only cares about signals on input connections. Instead of considering the system sequentially, we consider the system as a massively parallel system in which all modules processes signals concurrently. In Akka, all messages are passed to all \akkaActor{} instances and it becomes a design problem of filtering and producing data for other \akkaActor{} instances within an \akkaActor's system. The author considered all hardware modules as \akkaActor{} instances and in doing so, all modules can process information asynchronously within an Akka scheduler provided. Akka's schedulers may run sequentially or concurrently, it is decided by the scheduling algorithm but Akka forces developers to treat everything as asynchronous to improve performance and reduce changes required should an \akkaActor{} move to a distributed environment later \cite{TypesafeAkka2015}. While there is no intention of ever moving a hardware simulation to multiple distributed nodes, the concepts of immutable messaging and decoupling mutability and behaviour into \akkaActor{} instances reduces the development choices in a positive way. 

Akka's \akkaActor{} works by sending messages between \akkaActor{} instances. In the context of Akka, a message is an envelope around state. Most often, this is a \mintinline{scala}{case class} in Scala. As stated previously, a message has only immutable state. In the context of hardware simulation, a clock pulse might be represented as a message shown in \cref{lst:procsim-scala:clock-pulse}. This message only contains the time that the message occurred, represented as a \scalainline{Instant}\footnote{\scalainline{Instant} is part of the \scalainline{java.time} library: \url{https://docs.oracle.com/javase/8/docs/api/java/time/Instant.html}{}}. In order for another module to listen for this pulse, it must implement the \scalainline{Actor.receive} method which is a \scalainline{PartialFunction[Any, Unit]}. The \scalainline{def receive} method is usually implemented as a \scalainline{PartialFunction[]} using pattern matching \cite{Scala-PatternMatching} on message classes like \cref{lst:procsim-scala:clock-pulse}. \Cref{lst:procsim-scala:clock-pulse-actor} shows a \akkaActor{} that listens for our previously defined \scalainline{ClockPulse} message. In order to send a message to an \akkaActor{}, once an \akkaActor{} is created the \scalainline{!} operator sends the message: \scalainline{actor ! ClockPulse(Instant.now())}. With no other interaction required, the message is sent to \texttt{actor} regardless of where \texttt{actor} is found within the actor system. The simplicity of these interactions allowed for very straight-forward communication protocols between modules and the paradigms enforced by Akka eased development efforts for new modules directly improving development agility.

\begin{listing}[b!]
    \mint{Scala}|case class ClockPulse(time: Instant)|
    \caption{Clock pulse message for Akka}
    \label{lst:procsim-scala:clock-pulse}
\end{listing}

\begin{listing}[t!]
\begin{minted}{scala}
class ClockedActor extends Actor {
    override def receive = {
        case ClockPulse(time) => 
            println(s"Pulsed at $time")
        
        case _ => // do nothing
    }
}
\end{minted}
\caption{\akkaActor{} that listens for \scalainline{ClockPulse} messages and prints the time.}
\label{lst:procsim-scala:clock-pulse-actor}
\end{listing}


\section{Technology Challenges}
\label{sec:procsim-scala:technology-challenges}

This project progressed well while utilizing Akka on a local machine. However, in order to utilize Akka in the browser, much of Akka needed to be rewritten to utilize the browser concurrency semantics. A primary concern is that web browsers do not provide threading in the traditional operating system context, but they provide threading through the Web Workers specification \cite{MDN:WebWorkers}. Web Workers are not compatible with JVM threading as it is impossible to accomplish the full specification of JVM threads which include shared memory and sleepingunder the current specification standard \cite{MDN:WebWorkers, Doeraene2017}. As such, Scala.js has no means to implement Java's threading model natively within a JavaScript environment. Without a supported JVM threading library, Scala.js does not have the ability to compile Akka to run within the browser directly. Conversely JavaScript's programming model is heavily focused on asynchronous single-threaded programming paradigm through callbacks and surprisingly its model lends well to the actor model from \cite{Agha1985}. As \cite{Doeraene2014} eloquently explains:
\begin{displaycquote}{Doeraene2014}
    It may seem contradictory to implement an actor model, which is inherently concurrent and asynchronous,
    on a purely single-threaded platform like Scala.js. However, concurrency and asynchrony must not be confused with parallelism. While parallelism involves physically executing different tasks at the same time, e.g., on multiple processors or multiple machines, concurrency is a form of modularity which allows to model software components as independent units of execution and behavior which can communicate between each other.
\end{displaycquote}
\cite{Doeraene2014} implemented Scala Actors on top of the JavaScript VM utilizing a single-threaded ``asynchronous model'' as part of an Undergraduate project to port the original Scala actors implemented in \cite{Haller2009} to run on a JavaScript VM under the supervision of the authors of \cite{Haller2009}. Through further development, \cite{Doeraene2014} was extended to implement the Akka interfaces and became what is now known as Akka.js \cite{Stivan2015, akka-js2015}. The Akka.js project implements nearly all of the features of Akka encompassed entirely within a web browser JavaScript engine. The project has grown quickly over time, but at the time of our work it had large issues that we needed to manually patch in order to utilize Akka.js in a proper software stack. These complications created massive work flow problems and a majority of time was spent debugging Akka.js rather than working on the procsim.scala project itself.

In addition to its immaturity, Akka.js could not keep up with simulation requirements (\cref{req:simulations}) as it did not support small discrete times due to relying on JavaScript's \jsinline{Date} implementation's millisecond clock resolution and network latency \cite{MDN:Date}. Naively at this time, we intended accomplish real-time simulation speeds -- later realized to be a foolish goal. Given that we achieved message passing with approximately 50ms of latency between the sender and receiver, the latency and lack of precise clocks meant that real-time simulation was completely unattainable. Additionally, this resolution was not acceptable in the intended parallel simulation environment as multiple operations happened concurrently at any given moment of time. 

Another fault of ecosystem immaturity was that Scala.js created applications that were too heavy by default. At the time, the Scala.js compiler did not perform dead code elimination to an acceptable level leaving large applications to be distributed from a web server\footnote{As of July 2017, the Scala.js compiler has \textit{significantly} improved since the procsim.scala project and contains a powerful optimization tool-chain on top of Google's Closure Compiler \cite{Scala-js:CompOptPipeline}}. This meant that even small applications were in the order of mebibytes of minified JavaScript data to serve as part of an application compared to the kibibytes of information from other modern web frameworks. This size was unacceptable from an application distribution and usage stance (not adequately meeting \cref{req:personal}). 

\section{Analysis of Requirements}

Unfortunately, due to the immaturity of Scala.js, Akka.js and the ecosystem at the time, the procsim.scala project was ultimately discarded. The procsim.scala project had two contributions that should be addressed by future projects. First, using an ``always asynchronous'' approach to modelling components was very simple to reason about and eased implementation details significantly. Within Akka, all actors within an actor system are ``connected'' implying communication algorithms changed from a problem of ``where to send data?'' to ``which data does each module care about?'' This filtering was non-trivial and often meant that \akkaActor{} instances were spending more time filtering messages than performing actions. This author believes the question of ``which'' provides a simpler model than the direct connection model with the same power implying a need for future investigation. 

The second contribution is using a runtime DSL to configure the system gives students the ability to modify components behaviour similar to the hc12sim's instruction generation scheme discussed in \cref{sec:hc12sim:instruction-generation}. The massive benefit over the previous project was in runtime configuration instead of compile-time significantly reducing the required effort for students and providing a foundation for both pedagogical gains and configuration capabilities (\cref{req:pedagogical,,req:configuration}). As a qualification for any future DSL proposals, the DSL must feel familiar to students so that they do not need to learn entirely new syntaxes or paradigms (in the case of procsim.scala, this was using a VHDL-like syntax over Scala's typical syntax). 

Looking objectively at proscim.scala, it does not adequately meet many of the requirements outlined by \cref{sec:problem-statement}. Many of the following justifications are based on a theoretical product as it was not implemented to completion due to concerns raised in \cref{sec:procsim-scala:technology-challenges}. Utilizing procsim.scala within a web browser allows students significant ease of use on their personal computers meeting \cref{req:personal}. Though, regrettably due to the speed at which Akka.js performed in the JavaScript environment, its simulations were not fast enough to provide reasonable experiences making the simulations feel unresponsive. It is difficult to say how flexible the exposed interface for interacting with simulations would be without a full implementation. Most modern web application frameworks work heavily on asynchronous paradigms, thus they should be able to handle a slower model but may integrate poorly (\cref{req:modern}). \Cref{req:configuration} is aided by the use of the VHDL-like DSL language. For a pedagogical simulator procsim.scala is built to be simpler than a traditional fully behaviourally accurate simulator system. Lastly, procsim.scala was not completed far enough to showcase the functionality for simulation of modules for \cref{req:simulations}. One can assume that utilizing Akka actors, listening for signal messages would allow for collection of large amounts of state information.

\begin{table}[h!]
    \centering
    \begin{tabular}{l|cccccc}
        \textbf{Requirements} & \textbf{\hyperref[req:personal]{R1}} & \textbf{\hyperref[req:configuration]{R2}} & \textbf{\hyperref[req:pedagogical]{R3}} & \textbf{\hyperref[req:simulations]{R4}} & \textbf{\hyperref[req:modern]{R5}} & \textbf{Total} \\ \hline
        \textbf{\hyperref[ch:scala-akka]{procsim.scala}} & 
            3 & 4 & 3 & ? & 5 & \textbf{15} \\
    \end{tabular}
    \caption{Summary of requirement matching for procsim.scala.}
\end{table}

\chapter{Developing cross-platform C++ applications}
\label{ch:cross-platform}

\newcommand{\cmakeinline}[1]{\mintinline{CMake}{#1}}
    
After concluding work on procsim.scala, we applied lessons learned from procsim.scala and applied them to the original hc12sim project. Unfortunately, the hc12sim simulator was dated, poorly designed and poorly organized due to inexperienced student work. A large effort was placed in modernizing and correcting the hc12sim software to be well tested, build in a cross-platform environment, reduce feature development time and reduce the amount of time required for maintenance. The hc12sim project was renamed procsim as it was intended to be a general purpose simulation framework rather than a single purpose \hcmodel{} simulator. In an effort to improve modernize procsim, we designed and set up continuous integration infrastructure to effectively develop a new procsim suite. This chapter discusses changes required to an aged code base to bring support the C++14 standard, upgrading existing CMake \cite{Kitware:CMake} build script infrastructure and developing multi-platform environments for automated platform testing.

\section{Modernizing hc12sim}
\label{sec:cross-platform:sec:modernizing}

The hc12sim project that was showcased in \cite{Brightwell2013} was modern at the time of development. Since 2013, the C++ community has progressed thoroughly and many of the libraries in use within the project were superseded by the Standard Template Library (STL) on most compiler platforms. The hc12sim project utilized many poor practices such as excessive use of \cxxinline{typedef} statements, over use of threads or abuse of operator overloading facilities. hc12sim heavily relied on Boost's non-standard implementations of \verb|tr1| structures such as \cxxinline{boost::shared_ptr<T>} \cite{Boost1.53.0:SmartPointers} (now superseded by C++11's STL \cxxinline{std::shared_ptr<T>} \cite{cppreference:shared-ptr}) and the \cxxinline{boost::thread} with it's associated tooling \cite{Boost1.53.0:Thread} (superseded by C++11's STL \cxxinline{<thread>} \cite{cppreference:thread}). The insidious nature of threading and smart pointer usage throughout the hc12sim software made this effort a large task. When writing C++ software, distribution of binary files is a common concern when working in cross-platform environments. Thus a concious effort was made to remove all large binary-based libraries in favour of simple-to-build or header-only libraries to remove concerns of finding pre-installed development packages on multiple platforms. After replacing existing Boost implementations with STL equivalents, these changes needed to be cross-platform tested to validate that the implementations by different compiler vendors on varied operating system configurations behaved as expected.


\section{Cross-platform development}
\label{sec:cross-platform:sec:cross-platform-development}

The C++ language has unofficially always tried to follow the motto: ``write once, compile anywhere.'' With any compiled language, when moving between platforms it is often non-trivial to assure compiled code runs between any two platforms. Given that \cref{req:personal} requires any solution must run on student's personal computers, we required procsim run on the three major platforms, Microsoft Windows, macOS and GNU/Linux. Within the C++ ecosystem, the ubiquitous way to maintain cross-platform support is to utilize only commonly used tools and depend on the Standard Template Library (STL) as much as possible. That is not to say there are not powerful cross platform libraries (e.g. Boost \cite{Boost}, Qt \cite{Qt}), these libraries provide excellent tools but often introduce complex build requirements and add large binary files into any distributed application. An artefact of 2013, Microsoft's Visual C++ (MSVC) 2011 compiler and accompanying STL were not completely standards compliant with C++11 and thus to use modern C++11 STL features hc12sim required the use of Boost to get non-standard but similar implementations \cite{Microsoft:MSVC:ModernCPP:2011}. Easing refactoring efforts was many of the C++11 STL additions were modelled after Boost's implementations \cite{Meyers2005}. For procsim Boost's STL implementation dependency were mostly removed by upgrading the lowest supported compiler to MSVC 2015. MSVC 2015 fully supports C++11 in both compiler semantics and STL compatibility \cite{Microsoft:MSVC:ModernCPP}. This allowed us to fully remove the main Boost dependencies on Boost.Memory and Boost.Thread by effectively renaming the \cxxinline{boost::} namespace to use \cxxinline{std::} namespace for the components in \cxxinline{<memory>} and \cxxinline{<thread>}. By removing these library dependencies, it eased cross-platform development as only the Boost C++ headers were required to compile procsim's core library. However, the unit test platform for hc12sim was based on Boost.Test \cite{Boost1.53.0:Test} -- a problem addressed later.

\subsection{Access to multiple platforms}

Once procsim utilized the C++11 STL over Boost, the next task was configuring mechanisms to ensure each platform performed correctly. It was found multiple times that different compiler's STL implementations behaved differently depending on different versions and the compilation target in use. For example, we found that the GNU Compiler Collection (GCC) \cite{GCC} 5.X C++ STL shipped with a broken version of the C++11 \cxxinline{<regex>} library \cite{StackOverflow:GCCRegex}. But the GCC front-end found on macOS had no such issues as it uses \texttt{libc++} provided by LLVM with a GCC compliant compiler command-line interface that runs Xcode's version of LLVM clang \cite{Apple:Developer:CommandLineTools}. This bug was found well after feature implementation due to testing on Xcode 8.3 and MSVC 2015 passing successfully (our two main platforms). Building on Linux was tested and the issue was found through unit tests checking a feature that utilized regex searching functionality. This small but frustrating issue opened up a question: How can the software be verified working on three major platforms while not having native access to each platform? The original approach was to change physical devices to access the different major operating systems. For each machine, we downloaded the software from the development repository and ran tests manually. It became difficult to maintain build environment changes made over time to successfully build the project as little records were maintained and compiling system projects heavily depend on environment variables. This meant sharing the software development environment became difficult and near impossible to replicate. Further, if a developer did not have access to Apple hardware, macOS software could not be tested. In previous works, we utilized VMWare vSphere \cite{VMWare:vSphere} technologies to create common developer environments through Virtual Machines that could be used for consistent development and testing. 

Given that vSphere is a proprietary and expensive software, we researched other tools that provided virtualized environments. Oracle's VirtualBox \cite{Oracle:VirtualBox} provides virtual machine provisioning through open source tools. Building on the idea of ``cloning'' common configurations between development environments through VirtualBox, we eventually found HashiCorp's Vagrant \cite{VagrantUp}. Vagrant provides a consistent declarative configuration for defining a virtual machine environments through Ruby scripts. Vagrant's parent company, HashiCorp, hosts ``boxes'' that are preconfigured virtual machines ranging from Ubuntu Linux to FreeBSD to macOS El Capitain \cite{Vagrant:Boxes}. Boxes provide a base VM image that users configure to match the environment requirements for a project. When a configuration is settled, developers commit the configuration file, known as a Vagrantfile, to their repository to live directly beside the software being developed. The intention of storing environment configuration within software repositories is to attempt to merge ``build'' teams and development teams as these are no longer distinct processes. For example, \cref{lst:cross-platform:ubuntu-vagrant-box} show cases a simple configuration that installs both GCC 5.X and LLVM Clang \cite{LLVM:Clang} 3.7 -- two fully C++11 compliant compiler tool-chains into a Ubuntu Trusty 14.04 box. Due to Vagrantfiles being Ruby programs, they have full access to Ruby's gem ecosystem. We used Vagrant and VirtualBox to create three environments for Microsoft Windows 10, macOS El Capitain, Ubuntu 14.04 Precise and Ubuntu 16.04 Xenial. These configurations allowed instantiating virtual machines quickly and consistently on any platform to build and run test configurations. Vagrant allows users to create their own boxes and we created a box for each base configuration for distribution that would save time for setting up a machine. Our Vagrant box configurations are available at \url{https://github.com/Nava2/procsim-bld} available under the MIT Open-source License. With the three major platforms easily accessible, it became necessary to improve the build infrastructure to be consistent across these platforms.

\begin{listing}[tp]
\begin{minted}[breaklines=true]{Ruby}
Vagrant.configure("2") do |config|
    # Utilize the ubuntu/trusty64 box to give a 14.04 environment
    config.vm.box = "ubuntu/trusty64"
    
    config.vm.provider "virtualbox" do |vb|
        vb.cpus = 2
        vb.memory = 2048
    end
    
    # Install/update prerequisite software:
    config.vm.provision "shell", privileged: true, inline: <<-EOF
        echo "Installing build prequisisites: git, compiler ppas"
        apt-get install -q -y git 
        # Compiler tool-chains
        apt-add-repository -y ppa:ubuntu-toolchain-r/test 
        apt-add-repository -y ppa:adrozdoff/llvm-backport 
        apt-get update -qq
    EOF
    
    config.vm.provision "shell", privileged: true, inline: <<-EOF
        echo "Installing build compiler toolchains"
        apt-get install -q -y clang-3.7 \
                              gcc-5 g++-5 \
                              --force-yes
    EOF
    
    # Update all existing packages last
    config.vm.provision "shell", privileged: true, inline: "apt-get dist-upgrade -y ; apt-get autoremove --purge -y"
end # end vagrant file 
\end{minted}
\caption{Vagrant file that describes a Ubuntu 14.04 box with modern GCC 5.X and LLVM Clang 3.7 compilers installed.}
\label{lst:cross-platform:ubuntu-vagrant-box}
\end{listing} 

\subsection{Building with cross-platform tools}

CMake \cite{Kitware:CMake} is a cross-platform meta-build system. A meta-build system provides a ``front-end'' language for defining a project's build and generates build scripts for other build systems to consume. CMake provides support for many different build systems, each is supported through a CMake ``generator.'' The generators we chose to build on the Windows, macOS and Linux were ``Unix Makefiles'' for GNU \texttt{make} \cite{GNU:Make} on Linux and macOS, ``Xcode'' for macOS's IDE Xcode \cite{Apple:Xcode}, and ``Visual Studio 14 2015 [Win64]'' for MSVC 2015 \cite{CMake:Generators}. On Unix environments, \texttt{make} is generally available however and a newer build tool, \texttt{ninja} \cite{NinjaBuild}, is preferred as builds are faster and provides the same familiar behaviours as \texttt{make}. As stated, CMake provides a cross-platform meta-build system, CMake can not abstract direct interactions with compilers due to tool-chain and operating system specific options. For example, turning on optimization flags on GCC/clang differs from the MSVC platform. The original hc12sim project utilized CMake, but it was poorly organized and added large build crippling file dependencies. Changing a single file likely caused the entire build to trigger as if running a clean build rather than running an incremental build. At first, this was thought to be caused by CMake not properly structuring inter-file dependencies, however it was soon found to be due to poor header isolation and deep nested coupling. We separated all headers to try and isolate concerns of what a header defines -- often breaking headers into multiple smaller single purpose headers. These efforts attempted to remove inter-dependencies as much as possible. Reorganization of the procsim library coupled with improved build tools and the addition of \texttt{OBJECT} library files \cite{CMake:add_library} significantly reduced build time and allowed for less time spent compiling procsim. 

As previously discussed, compilers come with consequential differences in support for new and current standards. In order to try and utilize features appropriately it is often helpful to ``test'' compilers support of features. CMake provides this functionality through cmake-compiler-features \cite{CMake:compile-features} but CMake's built-in support lags heavily behind current C++ standards (including C++14 and 17 at the time of writing). Fortunately, compatibility \cite{CMakeCompatibility} is a library used to enhance compiler and STL feature detection within CMake projects. We worked with M\"uller to improve compatibility features and generation of header files that will include a ``shim'' on top of the current STL to add features where possible. ``Shims'' can only be added for STL features as syntax elements are not extensible in C++. To counter a lack of syntax tokens in some compilers, compatibility scripts also automatically generated C-style macros for keywords that are only available in certain C++ tool-chains (e.g. \cxxinline{constexpr} or \cxxinline{override}). These macros were used to add extra meta-information for the compiler where available to improve code generation. The use of compatibility allowed procsim to build more easily on the heterogeneous compiler platforms required to work on most personal computers for \cref{req:personal} while simultaneously improving the software's performance where possible.

\section{Testing Infrastructure}

Testing a multi-platform compiled systems project is very difficult and relies on many tools to be successful. Through utilization of Vagrant and better build infrastructure within CMake scripts, procsim was able to be automatically tested on multiple platforms through continuous integration software. Continuous integration describes a process of building, testing and deploying software as rapidly as possible to reduce time-to-market for software projects \cite{Stolberg2009}. Idealistically, by utilizing continuous integration, developers receive near-instantaneous feedback on whether or not a particular change set breaks the product in unforeseen ways -- this heavily depends on the level of testing and environments available. For procsim, we configured a small ``micro-PC'' with an Intel \textregistered{} i5-4590T and 4 GB of RAM to run a Jenkins \cite{Jenkins:Home} continuous integration server running Jenkins. The device was chosen due having CPU accelerated virtualization to aid in running Oracle VirtualBox through Intel VT-X.  

Within Jenkins, Vagrant was used to create a new virtual machine for each test platform which registered as ``slaves'' to the ``master'' Jenkins service provided by Jenkins' distributed build infrastructure \cite{Jenkins:DistributedBuilds}. Each virtual machine spawned from a separate base box to test procsim's software test suites consistently. For example, one machine spawned a Ubuntu 14.04 machine with manually installed compiler tool-chains and a second machine spawned a macOS instance with Xcode 8. Each machine ran the exact same unit tests after configuration of the virtual machine and selection of a CMake generator. This was achieved through a Jenkins Pipeline \cite{Jenkins:Pipeline} specified in a Jenkinsfile script \cite{Jenkins:Pipeline:Jenkinsfile}. Jenkins Pipelines allow for parallel execution of ``tasks,'' dependent and independent, for a continuous integration task graph. For procsim, due to its immaturity, the pipeline only included stages for building the software and running unit tests within the multiple operating system platforms required by \cref{req:personal} -- deployment tasks were unused due to project maturity. While not strictly related to simulation \cref{req:simulations}, the addition of a continuous integration solution allowed us to maintain a rapid pace of development and provide confidence that changes made worked across all platforms without time consuming manual confirmation.

Over the course of developing procsim, the build process became more and more complicated requiring improvements in automating build script writing. When writing unit tests, the original hc12sim project utilized Boost.Test \cite{Boost1.53.0:Test} for creating and running unit tests. Unfortunately, Boost.Test required library distribution which made working on platforms without official software distribution systems difficult (e.g. Windows or macOS). In order to remove the final thread of Boost binary dependence, we replaced Boost.Test with Catch \cite{CatchLib}. Catch is a header-only C++ testing framework built on top of modern C++. Catch provides a Behaviour-Driven Development (BDD) interface that allows developers to describe the behaviour of their system within their tests creating easily understandable and clearly organized tests \cite{Solis2011}. Catch improved the mechanisms used to test and provided an easier to maintain testing suite. Catch's test runner requires that all tests are defined through C-style macros and the executable must include Catch's main once per test suite through the \cxxinline{CATCH_CONFIG_MAIN} macro \cite{CatchLib:Tutorial}. When defining many tests, it is often useful to organize them in meaningful ``suites'' of test executables. With any testing framework, when creating multiple executables it is advantageous to share object resources to reduce compiling time -- particularly in \cxxinline{template} or \cxxinline{constexpr} heavy software which procsim rapidly became.

\section{Improving developer work flows within CMake}
\label{sec:cross-platform:sec:cmake-flows}

In \cref{sec:cross-platform:sec:cross-platform-development}, CMake was discussed as the build tool for procsim. CMake utilizes \texttt{OBJECT} libraries to allow multiple ``targets'' to share compiled object files and statically link against the same objects as required \cite{CMake:add_library}. These \texttt{OBJECT} libraries are not the same as shared objects or DLLs and should not be treated as such. An \texttt{OBJECT} library is a CMake abstraction on top of compilation units that must be statically linked into some other object. These shared \texttt{OBJECT} libraries were very useful in unit test files as a common object used between each executable ``suite'' is the \texttt{main} test runner. They provide a faster compiling but unportable binary object similar to a static library. Manually specifying all of these suite instances to get individual suites and manually maintain a global ``test all'' suite created significant time investment and manual adjustments whenever files were added or removed from the project. We developed a series of CMake macros and functions to reduce the required commands to define tests and library files in a ``project-based'' work flow \cite{CMake:macro, CMake:function}. Within CMake, any action that has a side-effect or result is considered a \texttt{target}. When CMake executes a build script, it creates a graph of inter-\texttt{target} dependencies. From the \texttt{target} graph, CMake calculates a topological sort for a dependency resolution scheme used to build the project. \Cref{lst:lua:cmake-test-declaration} shows a simple shared library defined utilizing the build infrastructure described. Three functions are used to define four distinct targets: 
\begin{enumerate}
    \item \cmakeinline{new_project()} defines a ``project'' that logically groups libraries, executables and tests;
    \item A library, \texttt{procsim} with headers, sources and external dependencies;
    \item Two test ``groups'' of logically packaged tests for the library from (1) and adds a dependency on (1) and a common test harness;
    \item An aggregate \texttt{PROJECT} suite that combines all the tests from each group into a suite for the current project
\end{enumerate}
Within these functions, each performs many tasks to try and simplify definition of build components to form a completed program build. This could include setting common compiler flags, introduce dependencies on vendor libraries or generating build-time code required for the target. 

\begin{listing}[hp]
\begin{minted}{CMake}
new_project()

# Define a new library called ``procsim''
new_library(procsim 
    HEADERS procsim/encoding/Algorithm.hpp
            procsim/encoding/Code.hpp
            procsim/encoding/Operand.hpp
            procsim/encoding/Primitives.hpp
            procsim/encoding/Utility.hpp
            
            procsim/time/Clock.hpp
            procsim/time/Timer.hpp
            procsim/time/AsyncTimerReceiver.hpp
    
    SOURCES encoding/Algorithm.cpp
            encoding/Code.cpp
            encoding/Operand.cpp
            encoding/Primitives.cpp
    
            time/Clock.cpp
            time/Timer.cpp
            time/AsyncTimerReceiver.cpp
    
    GENERATED_HEADERS ${PROCSIM_EXPORT_HEADER}
                      ${PROCSIM_COMPILER_HEADER}
    VENDORS cpp17_libs fmtlib                      # non-builtin
    INCLUDE_DIRECTORIES ${Boost_INCLUDE_DIRS}
    LIBS                ${CMAKE_THREAD_LIBS_INIT}) # builtin libs

# Create two test suites utilizing the new library      
create_test(time     SOURCES time/ClockTest.cpp 
                             time/TimerTests.cpp
                     LIBS    procsim test-harness)
                     
create_test(encoding SOURCES encoding/CodeTest.cpp
                             encoding/OperandTest.cpp
                             encoding/PrimitivesTest.cpp
                             encoding/UtilityTest.cpp
                     LIBS    procsim test-harness)
# Create a test for the whole project
create_test(core     PROJECT)
\end{minted}
\caption{CMake script showing how two test suites, \cmakeinline{time} and \cmakeinline{encoding}, are defined as part of a project, \cmakeinline{core}}
\label{lst:cross-platform:cmake-test-declaration}
\end{listing}
 
The function, \cmakeinline{new_project()} defines a set of ``global'' variables within the build that are named after the directory that the current defined CMake script is found. For example, if \cref{lst:cross-platform:cmake-test-declaration} was found in ``procsim/core'' the project would be called ``core.'' This behaviour is meant to be consistent and simplistic between project definitions. This project model is based on the idea of ``convention over configuration'' which was popularized by tools and frameworks like Apache Maven and Ruby on Rails \cite{Maven:StandardDirectoryLayout,Heinemeier:RailsDoctrine}. For example, \cmakeinline{new_project()} defines the following configurable global variables: 
\begin{itemize}
    \item \texttt{SOURCE\_DIR}: Implementation files directory, defaulting to ``core/src''
    \item \texttt{INCLUDE\_DIR}: Header include directory, defaulting to ``core/include''
    \item \texttt{TEST\_DIR}: Test implementation directory, defaulting to ``core/test''
\end{itemize}
The \texttt{*\_DIR} properties allow for inner-project commands to use relative paths to simplify declaration of their file dependencies. These properties are used to generate ``namespaced'' build targets and variables allowing for use in specification within CMake's dependency graph. To simplify, if you define two projects, \verb|a| and \verb|b|, you can set \verb|b| to depend on \verb|a| being built first. In \cref{lst:cross-platform:cmake-test-declaration}, all of the files are specified relative to the ``project directory'' removing repeated values required in standard CMake scripts. Once the \cmakeinline{new_project()} function is called within a cmake script hierarchy, all further functions from the build tools will utilize the state set by the project. 

The function \cmakeinline{new_library()} creates a new library target that compiles all of the specified files to create a library. The function accepts \texttt{SHARED}, \texttt{STATIC}, \texttt{OBJECT}, etc. to mimic \cmakeinline{add_library()}'s syntax \cite{CMake:add_library}  -- an attempt to remove some discontinuity between the two functions. As shown in \cref{lst:cross-platform:cmake-test-declaration}, there are several other parameters used: 
\begin{itemize}
    \item \texttt{HEADERS} Header files
    \item \texttt{SOURCES} Implementation files
    \item \texttt{GENERATED\_SOURCES} Source files that are generated at build time by another target
    \item \texttt{GENERATED\_HEADERS} Same as \texttt{GENERATED\_SOURCES} except for header files
    \item \texttt{VENDORS} These are external dependencies that are built within the build chain after downloading from an external source. This forces the library to depend on their download and build
    \item \texttt{LIBS} These are internal CMake libraries or other targets that are built in a multi-project configuration
\end{itemize}
Utilizing these parameters, the function creates an appropriate CMake \texttt{target} using \\* \cmakeinline{add_library()} and adds the required dependencies to the new library target. In addition, it performs platform-specific compiler adjustments to try and reduce the amount of configurations required per target. For example, many flags for MSVC to not apply to Clang which do not apply to GCC. This torrent of compiler-specific option configuration is eased by the use of CMake's generator expressions \cite{CMake:generator-expressions} but many compilers do not properly identify to CMake's internal infrastructure so manual adjustments must be made for newer tool-chains. By utilizing our own mechanism, the \texttt{target}s created have ``localised'' properties such as include directories and \cmakeinline{target_properties()} added in reproducible ways that do not accidentally pollute other targets or the global variable space. For example, if two targets require different versions of the same files, the built-in \cmakeinline{include_directories()} command could cause these targets to ``collide'' and fail to compile \cite{CMake:include-directories}. \cmakeinline{new_library()} appropriately applies the \texttt{SYSTEM} option to all includes that are outside of the multi-project build so any warnings generated are not applied. Lastly, \texttt{install} targets are created that install headers and compiled binaries to appropriate locations depending on the type of library and the platform in use. For example, on Microsoft Windows requires libraries be beside their executables, where UNIX utilizes PATH resolution mechanics.

The last function used in \cref{lst:cross-platform:cmake-test-declaration} is \cmakeinline{create_test()} which has three distinct incantations. The first creates a single test suite from Catch-based test sources \cite{CatchLib}. These sources are aggregated and compiled to an \texttt{OBJECT} library \cite{CMake:add_library}. The test objects are aggregated for future linking into the current \texttt{PROJECT}'s test executable. By generating \texttt{OBJECT} libraries, these smaller units of compilation can be combined and reused intelligently saving large amounts of compile time as previously discussed. The second pattern as \cmakeinline{create_test(test PROJECT)} generates a \texttt{PROJECT}-based test executable and ``check'' target that runs the \texttt{PROJECT} test executable. This test executable includes all previously defined \texttt{OBJECT} libraries in the current ``project.'' This executable is used to debug and execute tests for the given project, it is currently not possible to run a smaller test individually without utilizing the selected test runner's (Catch) command-line interface manually. Additionally, the generated executable is registered with CMake's CTest which creates build-script targets to run test executables providing dashboards and other useful tools \cite{CMake:CTest}. 

The last use case is \cmakeinline{create_test(target ALL)}, not included in \cref{lst:cross-platform:cmake-test-declaration}. The \texttt{ALL} request generates a test executable that includes every test suite created in a multi-project build into a single executable -- a convenience executable. As multi-project programs develop, it is useful to be more or less granular with the scope of tests executed depending on the context required at the time of use. While intuitively it appears as though large amounts of executables create large amounts of compilation units creating a very slow build time, due to the use of \texttt{OBJECT} libraries created at definition, adding more test executables only adds incremental link time due to linking static objects -- a marginal increase of time for the convenience provided. Summarized test output from procsim's generated \texttt{ctest} command is shown in \cref{lst:cross-platform:ctest-output}. The output is easy to read and, should errors occur, Catch provides assertion errors and suites will continue summarizing all errors after a suite completes.

\begin{listing}[hb!]
\begin{verbatim}
1>--- Build started: Project: RUN_TESTS, Configuration: Debug x64 ---
1>  Test project S:/research/procsim/build-windows
1>        Start  1: core-components
1>   1/11 Test  #1: core-components ............   Passed    0.09 sec
1>        Start  2: core-encoding
1>   2/11 Test  #2: core-encoding ..............   Passed    0.06 sec
1>        Start  3: core-time
1>   3/11 Test  #3: core-time ..................   Passed    4.46 sec
1>        Start  4: core-misc
1>   4/11 Test  #4: core-misc ..................   Passed    0.09 sec
1>        Start  5: conf-encoding
1>   5/11 Test  #5: conf-encoding ..............   Passed    0.20 sec
1>        Start  6: conf-loader
1>   6/11 Test  #6: conf-loader ................   Passed    0.06 sec
1>        Start  7: conf-arch
1>   7/11 Test  #7: conf-arch ..................   Passed    0.07 sec
1>        Start  8: conf-components
1>   8/11 Test  #8: conf-components ............   Passed    0.13 sec
1>        Start  9: conf-proc
1>   9/11 Test  #9: conf-proc ..................   Passed    0.18 sec
1>        Start 10: conf-time
1>  10/11 Test #10: conf-time ..................   Passed    0.07 sec
1>        Start 11: conf-env
1>  11/11 Test #11: conf-env ...................   Passed    0.04 sec
1>
1>  100% tests passed, 0 tests failed out of 11
1>
1>  Total Test time (real) =   5.50 sec
======= Build: 1 succeeded, 0 failed, 1 up-to-date, 0 skipped =======
\end{verbatim}
\caption{Test output from CTest \cite{CMake:CTest} from Microsoft Visual Studio Community 2015 for procsim.}
\label{lst:cross-platform:ctest-output}
\end{listing}  

All of these build improvements worked towards improving software maintainability of the simulation suite. Whilst these changes do not directly contribute to the Problem Statement in \cref{sec:problem-statement}, it is patently obvious that improvements in developer work flow have a direct consequence in improvements to developer time utilization and reduces the likelihood of poor software being produced. Given improvements to developer productivity, one may argue simply that this improves the validity of producing modern software (\cref{req:modern}) and any software produced will more accurately meet requirements specified \cite{Solis2011}. As such, these improvements allowed us to develop procsim more rapidly and reduced time spent fighting against compilers and cross-platform development issues.
\chapter{Lua-based configuration-driven processor simulation}
\label{ch:lua}

\newcommand{\luainline}[1]{\mintinline{Lua}{#1}}

 From working with Scala-based configurations in \cref{sec:sec:procsim-scala:configuration}, we believed that runtime-configuration was extremely important in encouraging students to try and manipulate processor designs. The author chose to expand the hc12sim project's the JSON-based, compile-time configuration capabilities replacing the mechanism with runtime-based configurations utilizing an embedded scripting language. Within the C++ programming ecosystem, there exists many different scripting environments that can be embedded with varying degrees of difficulty, feature capabilities and execution speeds. The author investigated utilization of several scripting environments before settling on utilizing Lua \cite{Lua:Homepage} with bindings provided through sol2, a wrapper between C++ and Lua API calls \cite{GitHub:ThePhD:sol2}. Once Lua was selected, the author designed configuration schemas built to execute in a Lua sandbox with timing and other hardware-level considerations abstracted away from the configuration definitions as much as possible.

\section{Utilizing runtime configurations through scripting}

procsim.scala had a VHDL-like syntax for defining instructions within a processor simulation. Providing the same facilities within a non-managed language like C++ is not possible without the use of a scripting engine. When researching possible solutions to this problem, several scripting languages stood out: Python \cite{Python:Homepage}, JavaScript through Google's V8 \cite{Google:V8} or Mozilla's SpiderMonkey \cite{MDN:SpiderMonkey}, Lua \cite{Lua:Homepage}, or reusing Scala through the Java Native Interface (JNI) \cite{Oracle:JNI}.

\subsection{Scripting language selection}

\subsubsection{Python} 

Several scripting languages were evaluated against each other for candidacy as the configuration specification language. First, Python was considered \cite{Python:Homepage}. Python is a mature, stable, dynamically typed language with wide use in industry, scientific and academic communities \cite{StackOverflowSurvey2016}. Python was considered due to it's exposed foreign function interface (FFI) which allows for C functions to be called from Python or C functions to call into Python. Python's syntax supports overriding operators \cite{Python:Operators}, object-oriented programming \cite{Python:Classes} and low-level bitwise operations \cite{Python:BuiltinTypes} -- features excellent for implementing low-level hardware simulations and configurations. For working with the Python VM, the Python community provides a library called CFFI that wraps python's FFI interface to call into and from Python. CFFI's purpose is to provide a bridge layer between Python and C easing the effort required. Any library wishing to utilize CFFI for interacting with the Python engine requires all exposed functions in the foreign library are externalized as C functions. procsim heavily employed C++11 syntax and does easily extend to a C-based API through its use of \cxxinline{template}, \cxxinline{inline} functions and \cxxinline{constexpr} values or functions. Thus porting procsim's API as a C API to utilize CFFI would become tedious. 

An alternate to the CFFI library is Boost.Python which provides binding mechanisms for working with complex classes in C++ and having them work within the Python virtual machine \cite{Boost1.53.0:Python}. Boost.Python provides a very simple and easy to use syntax for defining types. Further, Boost.Python respects \cxxinline{constructor} and \cxxinline{destructor} semantics of any types passed into Python. \Cref{lst:lua:python-example} showcases a simple type exposed into Python. Given the simplicity of exposing data into Python, and the ubiquitous nature of the language, it lends itself well to a pedagogical application (\cref{req:pedagogical}) and will feel modern to students coming from Software Engineering contexts (\cref{req:modern}). Regrettably, the largest compelling argument against Python is that to run Python scripts, one must install a Python virtual machine adding an extra dependency that is external to the project. This makes distribution more difficult for personal computers, though not all as some operating systems come with python pre-installed (e.g. most Linux distributions and macOS).

\begin{listing}[hp!]
\begin{minted}{C++}
// Simple Structure
struct World
{
    World(std::string msg): msg(msg) {} 
    void set(std::string msg) { this->msg = msg; }
    std::string greet() { return msg; }
    std::string msg;
};
\end{minted}

\begin{minted}{C++}
#include <boost/python.hpp>
using namespace boost::python;

// Define a python module
BOOST_PYTHON_MODULE(hello)
{
    class_<World>("World", init<std::string>())
        .def("greet", &World::greet)
        .def("set", &World::set)
        ;
}
\end{minted}

\begin{minted}{python}
>>> import hello
>>> planet = hello.World('hello')
>>> planet.greet()
'hello'
>>> planet.set('howdy')
>>> planet.greet()
'howdy'
\end{minted}
\caption{Example of exposing a C++ class to Python \cite{Boost1.53.0:Python}.}
\label{lst:lua:python-example}
\end{listing}

\subsubsection{JavaScript}

JavaScript is by far the most popular language in use in modern applications \cite{StackOverflowSurvey2016}. JavaScript provides many of the same features as Python, but does not have as ``strong'' of a type system. JavaScript provides many coercions of types into other types making it difficult to reason at times compared to Python (e.g. \mintinline{JavaScript}{{} + [] === 0} for reasons outside the scope of this document). While providing a modern interface for students (\cref{req:modern}), it also creates a regrettable pedagogical experience due to the extremely high-level nature of the language for low-level implementations (\cref{req:pedagogical}). In order to embed JavaScript within an application, it requires a JavaScript Virtual Machine to be embedded. The two current leading JavaScript engines are Google's V8 \cite{Google:V8} powering Google Chrome and Mozilla's SpiderMonkey \cite{MDN:SpiderMonkey} for Mozilla Firefox. SpiderMonkey provides a C++ API to embed functions and values into the engine. This C++ API is very similar to the API provided by Python's CFFI, however it provides slightly stronger type safety than a traditional C API utilizing \cxxinline{void*} arguments. When investigating to integrate the SpiderMonkey VM into procsim, an adverse design of SpiderMonkey is that the VM's entire state is bound to a single thread of execution \cite{MDN:SpiderMonkey:UserGuide}. JavaScript itself is a single-threaded language which is largely inconsequential for configuration declaration. Though due to SpiderMonkey itself being bound to a single thread, it may become too difficult to efficiently produce multi-threaded software when accessing SpiderMonkey JavaScript code at runtime. 

Google's V8 JavaScript engine is the most commonly used JavaScript engine as it powers Node.js, all Chromium-based browsers and Electron-based applications \cite{Google:V8}. Embedding V8 is a difficult task requiring large amounts of ``glue'' code to bind components \cite{Google:V8:Embedding}. Unlike SpiderMonkey, Simplified Wrapper Interface Generator (SWIG) provides a V8 wrapper \cite{SWIG:Homepage}. However, direct utilization of the V8 API is recommended. To use V8 in an application, it must be built and the library contains non-trivial build process per platform. Removal of non-trivial built libraries were removed as part of the work for \cref{sec:cross-platform:sec:modernizing}, thus adding a new complicated library proves contradictory to efforts previously made. To avoid the complicated build steps, for users to utilize V8 within procsim the entire engine would need to be shipped with the application or installed by users. Providing a working ``drop-in'' source for V8 or SpiderMonkey is unfortunately not feasible. As such, utilizing V8 or SpiderMonkey is not advantageous for procsim as an JavaScript is too high-level and the available engines are too large with large build dependencies.

\subsubsection{Java and Scala}

Java and Scala both require the Java Virtual Machine to execute software. Any application that has a JVM requirement involves either 1) packaging and shipping the entire JVM with the application or 2) expecting the user has a correctly installed JVM in default location. Both of these dependency resolution schemes for the JVM are difficult to complete with absolute certainty, but are not impossible. The JVM itself is also several hundred mebibytes in size with a large runtime overhead. We considered integration with the JVM because we had the existing DSL created within the procsim.scala project with a VHDL-like syntax that we could reuse. In order to access Scala or Java libraries outside the JVM, software wrappers must be written to utilize the Java Native Interface (JNI) \cite{Oracle:JNI}. The JNI is notoriously frustrating to write software for because C/C++ is not a managed language and the JVM is a managed VM making memory guarantees frustrating to correctly implement. Additionally, the JNI was developed to be efficient for software to communicate between two environments, it was not written to allow the process to be easy. Java developers are now discouraged from writing native libraries that interact with Java where possible as the JVM has significantly improved performance characteristics eliminating the largest use case for native libraries. SWIG provides an excellent wrapper definition tool to generate bindings for C++ classes into the JNI \cite{SWIG:Homepage} allowing developers to write generator files using a C++-like syntax which includes extra meta-information that is added to appease the Java Virtual Machine \cite{SWIG:Java}. SWIG uses ``directives'' to annotate existing C/C++ code to generate efficient JNI code that can then be compiled into a C/C++ application as any other software. SWIG does not support method references or lambda expressions which proved problematic when developing a communication layer. In addition, working with SWIG wrappers it non-trivial as classes become more complicated. SWIG has not updated over time to keep up with C++ standards and does not directly support many semantics such as \cxxinline{std::unique_ptr} and lambda expressions. With the additional overhead of utilizing the JVM, reusing the older procsim.scala DSL is not feasible as an embedded scripting language. Java is not designed to be embedded. Java was designed to be it's own managed runtime and have foreign native code be given selective access to run within the JVM. Due to large overhead costs at both build and runtime, we discarded Java as a viable scripting language.

\subsubsection{Lua}

Lastly, the Lua programming language was considered as it is by design an embedded scripting language \cite{Lua:Homepage}. The Lua Programming Language home page directly states: 
\begin{displaycquote}{Lua:Homepage}
    Lua is a powerful, efficient, lightweight, embeddable scripting language. It supports procedural programming, object-oriented programming, functional programming, data-driven programming, and data description.
\end{displaycquote}
\noindent These features fully encompass those features previously provided by Python while adding additional simplicity. Lua places data description as a first order citizen allowing for extremely descriptive dictionaries of heterogeneous data \cite{Ierusalimschy:PIL}. Similar to JavaScript's Object, Lua provides traditional object-oriented programming through an associative arrays known as tables \cite{MDN:Object, Ierusalimschy:PIL}. \Cref{lst:lua:project-example} shows a simple dictionary-like description of a ``project entry'' that the Lua community uses to display known projects that use Lua \cite{Lua:WhereIsLuaUsed}. Creating tables in Lua are a trivial task and through external libraries, such as tableshape, the schema or ``shape'' of a table can be quickly validated \cite{GitHub:leafto:tableshape}. 

Lua's design is first and foremost an embedded scripting language. Interaction with Lua's VM is through direct calls to Lua's C API. Many wrapper generation tools exist to allow higher-level binding than hand-written C API calls. SWIG provides a wrapper for Lua, but it has the same concerns as the Java generator discussed previously. For wrapping the procsim library, we considered sol \cite{GitHub:Rapptz:Sol} and luacppinterface \cite{GitHub:davidsiaw:luacppinterface} as they provided the best integration with C++ at the time. Both sol and luacppinterface provided tested interfaces for working with the Lua VM from C++, however sol provided more modern bindings for emerging C++ standards. Lua's syntax is very similar to other C-style languages and is commonly utilized in games world-wide showcasing the modern nature and performance characteristics of Lua (\cref{req:modern}). In addition to the traditional Lua engine, there is a hand-written JIT version of Lua known as LuaJIT. LuaJIT is extremely fast and provides machine-compiled Lua scripts at execution. Further, LuaJIT's C API is a superset of Lua's meaning it is fully compatible. 

Lua's syntax is extremely simple but provides flexibility with the ability to override all operators. This meant that the style of configuration intended for use with procsim.scala could be roughly ported to use within a Lua environment. When defining configurations, Lua is still a full programming environment with all of the power of a programming language adding a new layer to configuration capabilities for \cref{req:configuration}. Lastly, through LuaJIT, Lua is the near best performing scripting language after JavaScript. Lua provides a very simple interaction point thanks to wrapping libraries and Lua's embedding first design philosophy leaves it small, unbloated and easy to compile and package. 

\begin{listing}[t!]
\begin{minted}{Lua}
entry {
    title = "Tecgraf",
    org = "Computer Graphics Technology Group, PUC-Rio",
    url = "http://www.tecgraf.puc-rio.br/",
    contact = "Waldemar Celes",
    description = [[
        TeCGraf is Lua's birthplace,
        and the language has been used there since 1993.
        Currently, more than thirty programmers in TeCGraf use
        Lua regularly; they have written more than two hundred
        thousand lines of code, distributed among dozens of
        final products.]]
}
\end{minted}
\caption{Table used to describe project information for the Lua.org site \cite{Ierusalimschy:PIL}}
\label{lst:lua:project-example}
\end{listing}

\subsubsection{Scripting language decision}

After consideration of the four different scripting languages, it was narrowed to a choice between Lua and Python. Boost.Python's long history of use coupled with existing Boost usage in procsim made Python an enticing option. In addition, Python's feature set is very complete for most of the known use cases for procsim's configuration design requirements. A large issue with Python was that Boost.Python proved to be non-trivial to cross-compile on multiple platforms and imposed a direct coupling to the sizeable Python runtime. Further, to have simple configurations \mintinline{python}{dict} literals are required which provides less type safety guarantees than Lua's equivalent through tables validated with tableshape \cite{Python:BuiltinTypes, Ierusalimschy:PIL, GitHub:leafto:tableshape}. Lastly, Python's garbage collection can become difficult to work with in a non-managed language as it was not built with embedding in mind. Lua's focus on embedded scripting interfaces; its wide-spread use in games and industrial software as a scripting interface; the simplicity to build the C-based JIT or interpreted VM; and data description capabilities drove the decision to utilize Lua for configuration within procsim. 

\subsection{Lua integration}

Once Lua was chosen as a scripting language, the task of integrating the language into procsim's existing object model arose. As stated previously, there existed two main projects under consideration for wrapping C++ API into a scripting engine: luacppinterface and sol \cite{GitHub:davidsiaw:luacppinterface, GitHub:Rapptz:Sol}. When first attempting to implement a scripting interface, luacppinterface was chosen because it was simpler than sol. However due to the simple features available and the API design of luacppinterface, binding large \cxxinline{class} instances became overly complicated and runtime performance severely degratted. Thus, we migrated the software to sol. sol provided an elegant and still relatively simple C++11-based interface to wrap classes and functions to expose them to a Lua environment without the use of problematic C-style macros at zero runtime overhead cost \cite{GitHub:Rapptz:Sol}.

When working with sol, several bugs were found in the implementation of the wrapper. Of significant importance was in sol's current state, it could not pass a table to a C++ constructor \cite{GitHub:Rapptz:Sol:74} -- a feature we planned on utilizing heavily. This author raised the issue and after several weeks of remaining open, the author of sol never responded. However, another GitHub user by the name of ``ThePhD'' revived the project under a new name, sol2 (herein referred to as sol) \cite{GitHub:ThePhD:sol2}. sol was updated to include support for C++14 features and heavily relied on inline variadic template functions and SFINAE structures \cite{cppreference:SFINAE} to provide extremely efficient zero-cost bindings to Lua in a clean elegant syntax \cite{GitHub:ThePhD:sol2:benchmarks, GitHub:ThePhD:sol2:cxx-in-lua}. In \cref{lst:lua:sol2-example:player,lst:lua:sol2-example:usage,,lst:lua:sol2-example:bindings} sol2's capabilities are shown to simply bind the \cxxinline{class player} into Lua and utilize it within a simple toy script. Binding class properties is provided by utilizing \cxxinline{sol::property(...)} and \cxxinline{sol::readonly(...)} allowing a pattern very similar to Java Bean properties \cite{Oracle:JavaTutorial:JavaBeans}. The flexibility of overriding the operations allows for the creation of new ``syntax-like'' changes that alter the traditional behaviour of Lua operations as done with procsim.scala. While overloading operators is something heavily debated, the intention of changing ``typical'' behaviours is to reduce the amount of effort required by students. If the reading context remains consistent with traditional behaviour, a transparent side-effect should not increase the amount of knowledge required to use the system. The Java-bean style ``getter'' and ``setters'' let developers add or change side-effects of how an action works transparently to users. Without the wrappers provided by sol, this task is very difficult as it involves manually modifying Lua's meta-tables \cite{GitHub:ThePhD:sol2:usertype}.

\begin{listing}[hp]
\begin{minted}{C++}
class player {
public:
    int bullets;
    
    player(): player(3) 
    { }
    
    player(int ammo)
        : bullets(ammo), hp(10)
    { }
    
    bool shoot () {
        if (bullets < 1) 
            return false;
    
        --bullets;
        return true;
    }
    
    void set_hp(int value) {
        hp = value;
    }
    
    int get_hp() const {
        return hp;
    }
    
private:
    int hp;
};
\end{minted}
\caption{\cxxinline{class player} that holds two fields, one for hitpoints (\cxxinline{hp}) and one for bullets a player has (adapted from \cite{GitHub:ThePhD:sol2:cxx-in-lua}).}
\label{lst:lua:sol2-example:player}
\end{listing}

\begin{listing}[hp]
\begin{minted}{Lua}
-- player_script.lua

-- call single argument integer constructor
p1 = player.new(2)

-- p2 is still here from being
-- set with lua["p2"] = std::make_shared<player>(0);
-- in cpp file
local p2shoots = p2:shoot()
assert(not p2shoots) -- had 0 ammo

-- set variable property setter
p1.hp = 545;
-- get variable through property getter
print(p1.hp);

local did_shoot_1 = p1:shoot()
print(did_shoot_1)
print(p1.bullets)
local did_shoot_2 = p1:shoot()
print(did_shoot_2)
print(p1.bullets)
local did_shoot_3 = p1:shoot()
print(did_shoot_3)

-- can read
print(p1.bullets)
-- would error: is a readonly variable, cannot write
-- p1.bullets = 20
\end{minted}
\caption{Utilize the \cxxinline{class player} within a Lua script (adapted from \cite{GitHub:ThePhD:sol2:cxx-in-lua}).}
\label{lst:lua:sol2-example:usage}
\end{listing}

\begin{listing}[hp]
\begin{minted}{C++}
#include <sol.hpp>

int main () {
    sol::state lua;
    
    // Register usertype metatable
    lua.new_usertype<player>( "player",
        // 2 constructors
        sol::constructors<player(), player(int)>(),
        
        // member function that returns a variable
        "shoot", &player::shoot,
        
        // gets or set the value using member variable syntax
        "hp", sol::property(&player::get_hp, &player::set_hp),
        
        // can only read from, not write to
        "bullets", sol::readonly( &player::bullets )
    );
    
    // set a variable "p2" with a new "player" with 0 ammo
    // using an std::shared_ptr<>
    lua["p2"] = std::make_shared<player>(0);
    
    // run the example script
    lua.script_file("player_script.lua");
    
    const std::shared_ptr<player> p1 = lua["p1"];
    std::cout << p1->hp << std::endl; // prints 545
}
\end{minted}
\caption{Required bindings to allow utilization of \cxxinline{class player} from Lua (adapted from \cite{GitHub:ThePhD:sol2:cxx-in-lua}).}
\label{lst:lua:sol2-example:bindings}
\end{listing}

Working with sol was not entirely without problems. Because sol was actively developing, we worked closely with the new author of sol, ``ThePhD,'' to implement features and verify behaviours across multiple platforms (a requirement deeply discussed within \cref{sec:cross-platform:sec:cross-platform-development}). We implemented tests and benchmarks on the macOS platform as ``ThePhD'' did not have access to the platform. We worked to add continuous integration support for sol through Travis-CI \cite{TravisCI} to confirm the libraries support on platforms required for procsim \cite{GitHub:ThePhD:sol2:pr:17, GitHub:ThePhD:sol2:pr:18}. The contributions surrounding continuous integration improvements were re-purposed configurations from procsim provided as open-source contribution to sol. Within procsim, the development heavily tracked the cutting-edge of sol development. By tracking sol closely, performance fixes and improvements were rapidly integrated into our implementation. Problematic ergonomics were discussed and often implemented and a symbiotic relationship was built that benefited implementation of procsim and sol. 

\subsection{Configuration versus Simulation entities}
\label{sec:lua:sec:configuration-vs-simulation}

The original hc12sim project shared all modules between both configuration and simulation. This meant less lines of code to understand and generally made the software ``easier'' to write for novice developers. However as modules grow in size, they become more complicated to maintain. Something found in work for \cref{ch:scala-akka} was that immutable objects produce easier to reason and better performing software. In developing configuration entities to specify simulation entities, it was found that a lot of meta-information required to make configurations easier to specify could be removed once the system was fully specified. The meta-information was provided to students to reduce their cognitive load but does not provide value to a simulation. Most values within the configuration entity need to be mutable within a configuration file, but within the simulation these values are immutable allowing for simpler testing of a simulation engine.

For example, when defining a \texttt{Register} component, there are a several values that are required to when configuring an instance: 
\begin{itemize}
\item name - A logical name
\item clock - The clock that this Register is bound to (Registers are sequential)
\item readCount - Number of clock ticks this takes to read a value (required to ``slow'' down execution)
\item writeCount - Number of clock ticks to write a value to the Register
\item access - What type of access this register supports, i.e. read or write
\end{itemize}
When writing a configuration, the it is easier to define a \texttt{clock} by its name, a literal definition or by utilizing a \luainline{local} value. By contrast, at simulation time the \texttt{clock} must be a reference to an existing instance from somewhere. This idea of ``higher-level configuration'' that is transformed into a ``lower-level'' representation directly mimics a compiler's approach when compiling high-level code to the compiler's intermediate representation. The intermediate representation removes superfluous information in exchange for a smaller but easier to optimize representation of the programs state which is how simulation entities are represented. We have included most of the class definitions for simulation entities in \cref{ch:procsim-sources}.

% We removed half of these parameters from the ts version -- they have no use when you are treating everything
% as I/O connections, they just aren't necessary. Want a read-only register? Don't connect something to D.


\section{Lua-based Configuration}

To limit the scope of a system's configuration, procsim only exposes the following components: registers, memory banks (ROM, RAM), clocks, and arithmetic operations. Simplification of the model allows for reduction of complexity in an attempt to reduce the cognitive load for students when creating their own processors, similar to the approach taken by \cite{Skrien2001} and \cite{Garcia2009}. This trade-off removes excessive modules in favour of simplifying interfaces to improve pedagogical outcomes (\cref{req:pedagogical}). The minimal canonical example of a Turing-complete machine is URISC, popularized by \cite{Mavaddat1988} and \cref{fig:urisc-architecture} shows the hardware architecture of URSIC which \cref{lst:lua:urisc-example-1,lst:lua:urisc-example-2} was based off of which we use as a referential base for configuration specifications.

\begin{figure}[bp!]
    \centering
    \includegraphics[width=0.5\linewidth]{img/urisc-architecture}
    \caption{URISC hardware architecture with microcodes \cite{Mavaddat1988}.}
    \label{fig:urisc-architecture}
\end{figure}

\begin{listing}[hp!]
    \inputminted[escapeinside=||, lastline=42]{lua}{./listings/urisc.lua}
    \caption{Configuration of \cref{fig:urisc-architecture} for the URISC processor \cite{Mavaddat1988}.}
    \label{lst:lua:urisc-example-1}
\end{listing}

\begin{listing*}[hp!]
    \inputminted[escapeinside=||, firstline=45]{lua}{./listings/urisc.lua}
    \caption{(Continued) Configuration of \cref{fig:urisc-architecture} for the URISC processor \cite{Mavaddat1988}.}
    \label{lst:lua:urisc-example-2}
\end{listing*}

Because procsim's configurations are written in Lua, they are fully Lua compliant scripts in every aspect. In \cref{lst:lua:urisc-example-1} on line 4, a local variable is defined that is used throughout the processor definition. While defining a numerical value of four is trivial, this potential expands to allow for definitions that include loops and other control flow statements to simplify definitions of large or iteratively built module configurations. This mechanic defines why these configurations are so powerful, the example of a variable declaration is a shamefully inadequate description. The idea to utilize Lua as a means of ``configuration as code'' was inspired by previous work with Google Guice through its \javainline{Module<T>} syntax \cite{Google:Guice:Motiviation} and annotation driven serialization with Jackson Annotations \cite{GitHub:Jackson:Annotations}.

We use the Lua scripting engine to directly execute a specified configuration as a Lua script. When the script executes, it must create and return a \luainline{Proc} instance. While the script executes, any instantiations of types are projected through Lua into the C++ configuration entities as described in \cref{sec:lua:sec:configuration-vs-simulation}. These entities have properties that can be easily read and reused within a configuration script. Once the full configuration is returned, the \luainline{Proc} and it's children must be converted into near-immutable simulation entities. We have included the definitions of the simulation classes within \cref{ch:procsim-sources}. We elaborate on how each of these configuration entities are defined within configurations in further discussions.

Moving further into the URISC example in \cref{lst:lua:urisc-example-1,lst:lua:urisc-example-2}, all objects use a static \luainline{new} method that accepts a Lua table of configuration values. This approach was utilized to mimic JavaScript's colloquially named ``option object'' of parameters to make parameter values as obvious as possible by using explicit keys. For each table-based function, the keys and values are validated against an expected schema -- invalid or missing keys or bad value types throw informative exceptions for students to recover from. All instances within the configuration schema inherit from \cxxinline{class ConfObj} which defines that every instance must have a name. These names are utilized to lookup references in other referenced modules. From these names, the ``compilation'' of the configuration utilizes a two-phase compilation pattern inspired by a typical assembler pattern. The first phase runs across the entire structure and creates a list of required references attempting to verify any named reference has an associated configuration object. This table of references are sorted using a topological ordering such that every module is built before it is required. Each configuration value is converted through sol2 bindings to create non-configuration entities 

\paragraph{Clock.} Breaking down each component type, first is definition of a \luainline{Clock}, repeated in \cref{lst:lua:urisc:clock}. \luainline{Clock} modules must have a \luainline{:period} or \luainline{:frequency}\footnote{In Lua, the object method invocation syntax is: \luainline{object:method} which passes \texttt{object} as the \texttt{this} parameter. For the remainder of this document, the \texttt{this} parameter is omitted for brevity and is implied when prefaced by \texttt{:}.} -- defining one implies the other. These each have useful ``helper'' functions that allow for defining relative values. For \luainline{:period}, the helper functions: \luainline{seconds()}, \luainline{millis()}, \luainline{micros()} and \luainline{nanos()} define times relative to nanoseconds. For \luainline{:frequency} the style of functions exist based on Hertz (e.g. \luainline{MHz()}, or \luainline{KHz()}). These \luainline{Clock} modules generate instances that produce ``tick'' operations based on the \luainline{:period} and allow synchronous modules to listen for events. 

\begin{listing}[h!]
    \inputminted[escapeinside=||, firstline=10, lastline=13]{lua}{./listings/urisc.lua}
    \caption{\luainline{Clock} configuration for the processors main clock (cut from \cref{lst:lua:urisc-example-1}).}
    \label{lst:lua:urisc:clock}
\end{listing}

\paragraph{Register.} In procsim, a \luainline{Register} is a simple memory unit that stores a single value. The \luainline{PC} definition is shown in \cref{lst:lua:urisc:register}. The \luainline{:width} parameter represents the bit-width of the \luainline{Register}'s stored value. \luainline{access} is a set of enumeration values that are used to define ``getter'' and ``getter'' methods for the ``value'' of a \luainline{Register}. If the \luainline{:access} parameter is \luainline{Access.ReadWrite} then the value may be set or read. Alternately if the \luainline{access} for an example \luainline{Register} \luainline{R} is \luainline{Access.Write} then the following instruction code will not compile in the Lua interpreter: \luainline{local r_plus_1 = proc.R + 1}. By setting the \luainline{:access} to \luainline{Access.Write} the value can not be read from the register -- effectively disconnecting the \verb|Q| signal in traditional \luainline{Register} diagrams. procsim achieves this behaviour by not binding a ``setter'' or any arithmetic operations for the \luainline{Register} type bound through sol to the property for \luainline{proc.R} parameter utilized when defining an \luainline{Instruction} execution (see \cref{sec:lua:sec:instructions:sec:exec}). The \luainline{:readCount} and \luainline{:writeCount} parameters define how many clock ``ticks'' it takes to read or write to the \luainline{Register} respectively. These values are utilized by the execution engine to schedule the triggering of events.

\begin{listing}[th!]
    \inputminted[escapeinside=||, firstline=17, lastline=23]{lua}{./listings/urisc.lua}
    \caption{\luainline{Register} configuration for the program counter (cut from \cref{lst:lua:urisc-example-1}).}
    \label{lst:lua:urisc:register}
\end{listing}

\paragraph{Memory.} The second last module type available are \luainline{Memory} units. \luainline{Memory} are either read-only or read-write like \luainline{Register} values. \Cref{lst:lua:urisc:memory} shows the RAM definition for the URISC architecture. Parameters are identical to a \luainline{Register} except that they additionally have a \luainline{:length} parameter that defines the length of a \luainline{Memory} in units of \luainline{:width} (the word width). In the sample, the memory is 16KiB ``words'' wide. With a RAM, depending on the type of memory being modelled, it is often useful to have the \luainline{:clock} parameter be slower than the processor clock. This models the exponentially different time taken to read from a \luainline{Register} rather than cache or off-chip memory.

\begin{listing}[h!]
    \inputminted[escapeinside=||, firstline=35, lastline=42]{lua}{./listings/urisc.lua}
    \caption{Memory unit configuration for the program counter (cut from \cref{lst:lua:urisc-example-1}).}
    \label{lst:lua:urisc:memory}
\end{listing}

\paragraph{Proc.} The second last module defined is the processor itself through the \luainline{Proc.new()} function. The \luainline{:clock} parameter defines the clock utilized by the processor itself. This value is not shared amongst internal modules unless explicitly named by another module. The \luainline{:registers} table defines \luainline{Register} values that live ``on-chip'' within the created \luainline{Proc}. The \luainline{:memory} table defines a sparse-matrix representation of the processor's memory mapping. Any inline defined \luainline{Memory} or \luainline{Register} components are given the name of the key in the table they are specified in. This is a small convenience for students to not replicate information which risks typographic mistakes. Within the \luainline{:memory} map, any memory unit may be defined at an ``address'' specified. In the URISC machine, there are no memory mapped registers, however \cref{lst:lua:memory-mapped} displays a small processor with three \luainline{Register} values mapped to the first three addresses and a RAM mapped starting at address \luainline{0x0004}. Specification of \luainline{:memory} mappings take into account the width of the memory elements specified and will do partial reads or writes depending on what is required. The current implementation does not support memory elements smaller than the word size. The \luainline{:instructions} table provides a look-up table of instruction information including execution patterns for an \luainline{Instruction}. \luainline{Instruction} definitions are discussed in the next section, \cref{sec:lua:sec:instructions}.

\begin{listing}[hp!]
    \inputminted[escapeinside=||]{lua}{./listings/memory-mapping.lua}
    \caption{Processor utilizing memory mapping functionality.}
    \label{lst:lua:memory-mapped}
\end{listing}

\section{Instruction definition}
\label{sec:lua:sec:instructions}

The \luainline{Instruction} definition is more complicated than any other module and where the best benefits are found from a programmatic configuration. Within the \luainline{proc:instructions} table, the \luainline{:name} of the \luainline{Instruction} is calculated based on the value of the key an \luainline{Instruction} is assigned to, mimicking the \luainline{proc:registers} table. 

\subsection{Opcode and Operand encoding}

procsim allows for easily defining complex encodings for opcode and operand values. \\* \luainline{Instruction:code} specifies the opcode of an \luainline{Instruction}. A \luainline{Code} definition describes values encoded within an opcode a traditional microarchitecture. \Cref{lst:lua:urisc:encoding} shows a cut of the encoding definition for the URISC \texttt{SUBLEQ} \luainline{Instruction}. Within the \luainline{:code} parameter, the \luainline{Code} construct defines an opcode. The first parameter of \luainline{Code}'s constructor is a constant value with bits that remain constant in all encodings of the current instruction -- both high and low values are. The second parameter is a table of ``fields'' that are encoded within the opcode when compiled. Any key passed into the table will be extracted from the opcode at decoding time and passed into the execution handler attached as an ``operand field.'' With both the constant opcode and the table of fields, \luainline{Code.new{}} calculates the constant bits of the opcode encoding. Once all \luainline{Instruction} values are defined for a \luainline{Proc}, the configuration engine computes a discriminant decoding schema for runtime. When defining a field, there are several helper functions in use. Line 51 of \cref{lst:lua:urisc:encoding} includes the helper function, \luainline{u16(15)} which creates an unsigned 16-bit field at index 15\footnote{While Lua uses a 1-based indexing scheme like MATLAB or FORTRAN, procsim uses zero-based indexes to match C.}--indexed from the least-significant bit to the most-significant bit. The naming scheme for the primitive field helper functions are named using the Rust programming language's names for primitive types as they are short and unambiguous in their context \cite{Rust:PrimitiveTypes:Numeric}.

\begin{listing}[hb!]
    \inputminted[escapeinside=||, firstline=48, lastline=59]{lua}{./listings/urisc.lua}
    \caption{Encoding of the opcode and operands of the \texttt{SUBLEQ} \luainline{Instruction} (cut from \cref{lst:lua:urisc-example-1}).}
    \label{lst:lua:urisc:encoding}
\end{listing}

The decoding scheme for a processor specification is computed from the table of \\* \luainline{Instruction}s in \luainline{Proc:instructions} by computing individual discriminants per instruction and then cross-checking each discriminant against other instructions in a pair-wise fashion. To check if two discriminants are ambiguous, mask each ``constant'' opcode with another's ``constant mask'' and vice versa and if the resultant values are equal, the constant values are equal implying there is an ambiguous opcode specified. \Cref{alg:lua:opcode-verification} details the full computation used to verify ambiguous opcode specifications. If an ambiguous opcode is found within the instruction table an exception is thrown alerting the user which instructions have ambiguous conflicts and which constant bit values are shared. Unfortunately, this verification algorithm operates in a $\text{O}(n^2)$ complexity as each operation must be compared to another. A solution utilizing a radix-based sort could reduce the search space for each constant operand to make the algorithm $\text{O}(n \log n)$ \cite{Goodrich2014} -- though the complexity of each step within this computation is of trivial time due to consisting of two bitwise operations, making the trade-off between readable and fast code inconsequential. This specification scheme for encoding allows for extremely flexible ISA design but also provides a mechanism to allow students to find and recover from mistakes. By using this simplistic specification system and providing feedback for students, students will have an easier time learning why certain ISA design schemes do not work. 


\begin{algorithm}[hp!]
    \caption{ISA opcode verification algorithm to find ambiguous opcode definitions}
    \label{alg:lua:opcode-verification}
    \begin{algorithmic}[1]
        \Statex \textbf{Input}: $ins$ = Array of Instruction definitions
        \Statex \textbf{Output}: Array of Instruction definition pairs with ambiguous opcode values.
        
        \Statex \textbf{Step 1: \emph{Compute a discriminant mask for each Instruction}}
        
        \State Define a buffer for computed discriminants and masks
        \State $insBuffer \gets \left[\right]$
        \ForAll{$i \in ins$}
            
            \State Build a mask for field location in the opcode
            \State $fieldMask \gets 0$
            \\
            \ForAll{$field \in i.fields$}
                \State Compute the field's bit mask
                \State $currentMask \gets \left( \text{mask}\left( field.width \right) \ll field.index \right)$
                \\
                \State Mask the current field's mask with the overall field mask
                \State $fieldMask \gets field.mask \mathrel{|} currentMask$
            \EndFor
            \\
            \State Compute the constant bits of the opcode, bits that are not part of fields
            \State $opcodeMask \gets \textbf{not}~fieldMask$
            \State $discriminant \gets ins.opcode$
            \\
            \State Store the discriminant and mask
            \State $insBuffer.push\left( \left[ins, disciminant, opcodeMask \right] \right)$
        \EndFor
        
        \Statex\textbf{Step 2: \emph{Use the computed masks to search for ambiguities}}
        \State $ambiguous \gets \left[\right]$
        \State Perform a cross-validation of Instruction values
        \ForAll{$[computed1,~index] \in insBuffer$ }
            \ForAll{$computed2 \in insBuffer\left[index+1:\right]$}
                
                \State Unpack the computed values
                \State $[ins1,~ discriminant1,~ opcodeMask1] \gets computed1$
                \State $[ins2,~ discriminant2,~ opcodeMask2] \gets computed2$
                \\
                \State Compute a combined opcode mask
                \State $cMask \gets opcodeMask1 \mathrel{\&} opcodeMask2$
                \If{$\left(discriminant1 \mathrel{\&} cMask\right) = \left(discriminant2 \mathrel{\&} cMask\right)$}
                    \State The two opcodes are ambiguous if equal
                    \State $ambiguous.push\left(\left[ins1, ins2, cMask\right]\right)$
                \EndIf
            \EndFor
        \EndFor
        
        \Return{ $ambiguous$ }
    \end{algorithmic}
\end{algorithm}

Operands for an \luainline{Instruction} are specified within the \luainline{:op} parameter table. \luainline{:op} accepts a table of key to field specifications that behaves identically to the second parameter of \luainline{Code.new()}. Unlike the \luainline{:code} parameter, there is no verification for operand values. Keys within the \luainline{:code} and \luainline{:op} tables are checked for uniqueness and Lua failed to compile if a table has non-unique keys. The table values are unioned and their widths are computed, storing the total size of the operand for encoding into machine code. As with \luainline{:code}, the keys are extracted and from the compiled encodings and passed to the execution function. 

\subsection{Instruction execution: Side-effect-based compilation}
\label{sec:lua:sec:instructions:sec:exec}

The intention of the \luainline{Instruction:exec()} function is to provide the schema for how an \luainline{Instruction} executes within the context of a simulated processor and operands passed. When students provide a \luainline{:exec()} method, they are not writing the function that will directly execute within the engine. procsim.scala discussed utilizing an event-based execution schema that behaved through message passing; the \luainline{:exec()} method describes a sequence of events that will register within an event queue when the \luainline{Instruction} is called. When defining the syntax, we attempted to simplify the logic behind defining \luainline{Instructions} by converting normal Lua-operations (e.g. arithmetic) into sequences of events instead. \Cref{lst:lua:urisc:exec} defines the execution of the very simple \texttt{SUBLEQ} instruction used in the URISC architecture. Executing the \luainline{:exec()} function performs a process akin to compilation using the state changes defined by the function itself and utilizing side-effect operations to ``collect'' the sequence of state changes within the function definition. This process allows the engine to compile the execution events through a sequential mechanism and create a state machine that behaves as a student describes it. This does not provide a fully-asynchronous behaviour like VHDL's connection syntax, but does allow an execution engine to decide how state changes are scheduled. The following sub-sections describe the Lua statements found within \cref{lst:lua:urisc:exec} and how they compile to scheduled events within procsim.  

\begin{listing}[h!]
    \inputminted[escapeinside=||, firstline=61, lastline=74]{lua}{./listings/urisc.lua}
    \caption{Execution definition for the \texttt{SUBLEQ} \luainline{Instruction} (cut from \cref{lst:lua:urisc-example-1}).}
    \label{lst:lua:urisc:exec}
\end{listing}

\subsubsection*{Assignment}
\label{sec:lua:sec:instructions:sec:exec:sec:Assignment}

An assignment statement, e.g. \luainline{proc.R = operand.A} is scheduled using \cref{alg:lua:assignment}. The right-hand side arguments is known as the \texttt{rval} which is treated as an rvalue in C++ terminology \cite{cppreference:ValueCategories}. The left-hand size argument is the value assigned to and is known as \texttt{lval} mimicking lvalue in C++ \cite{cppreference:ValueCategories}. When working with assignments, this event-based architecture allows a simulation engine to reorder assignments depending on how the execution will progress. This flexibility allows for implementations of optimization patterns at a future time. 

\begin{algorithm}[h!]
    \caption{Assignment statement compilation}
    \label{alg:lua:assignment}
    \begin{algorithmic}[1]
        \Statex \textbf{Input}: $rval$ = \texttt{rvalue} read from 
        \Statex \textbf{Input}: $lval$ = \texttt{lvalue} assigned to
        \Statex \textbf{Input}: $bus$ = Bus for data path
        \Statex \textbf{Side-effect}: Bus contains the value from $rval$
        \Statex% blank
        \If{$rval$ is an operand}
            \State Read the bit pattern from the decoded operands
            \State $bus \gets operand$
        \Else 
            \State Value is a module within the machine
            \State $bus \gets rval$
        \EndIf
        \\
        \State $lval \gets bus$
    \end{algorithmic}
\end{algorithm}

%\begin{enumerate}
%    \item Read the rvalue \cite{cppreference:ValueCategories} of the assignment onto a BUS
%    \begin{itemize}
%        \item if the rvalue is an operand, this value is read from the decoding sequence and written to the lvalue \cite{cppreference:ValueCategories} location.
%        \item Assignment statements may be rewritten depending on order of encoding operations if and only if the assignment has no side-effects in the state machine
%    \end{itemize}
%    \item Write the value from the BUS into the lvalue specified (i.e. \luainline{proc.R})
%\end{enumerate}

\noindent Lua's \luainline{local} variables are handled utilizing the result of any operations or assignments as a ``BusValue'' representing the state of the Bus in use rather than traditional \texttt{integer} values. For example, after an assignment, the result of an operation is a ``BusValue'' that states what is currently on the data bus if and only if the assignment is not a direct module lookup. For example, consider the following \luainline{local} assignments: 

\begin{listing}[h!]
\begin{minted}{lua}
-- Create an alias, B is proc:B
local B = proc:B

-- Create a `BusValue' that has the value of the operation B + A
local C = proc:B + proc:A

-- Use the value on the databus as the address into memory
-- And store the value of that address in that location
proc:memory[C] = C -- 'result' of this is a `BusValue' of C, still
proc:X = proc:A + proc:B -- 'result' of this is a `BusValue' 
                         -- with value proc:X
\end{minted}
\caption{Local variable definitions within an \luainline{Instruction:exec()} method.}
\end{listing}

\noindent Implicitly created ``BusValue'' semantics allow for code reuse within implementation definitions while remaining behaviourally accurate. \luainline{local} defined variables provide a simple way for students to write ``high-level'' code to simplify their implementations.

\subsubsection*{Memory Access}

Memory accesses, \luainline{proc.memory[B]}, read a memory location or write to it depending on whether the access is an lvalue or rvalue. In \cref{lst:lua:urisc:exec} on line 65, the memory access is used as an lvalue, and we utilize \cref{alg:lua:memory-access:read} to schedule the events.

\begin{algorithm}[h!]
    \caption{Memory read compilation}
    \label{alg:lua:memory-access:read}
    \begin{algorithmic}[1]
        \Statex \textbf{Input}: $memory$ = Memory unit to read from or write to
        \Statex \textbf{Input}: $address$ = address within $memory$
        \Statex \textbf{Input}: $abus$ = Bus for address path
        \Statex \textbf{Input}: $dbus$ = Bus for data path
        \Statex \textbf{Side-effect}: $abus$ contains $address$
        \Statex \textbf{Side-effect}: $dbus$ contains the value from $memory\left[address\right]$
        \Statex% blank
        \Statex Write the $address$ to $abus$
        \State $abus \gets address$
        \Statex Set $memory.readWrite$ to $READ$
        \State $memory.readWrite \gets READ$
        \\
        \Statex Memory buffer register gets the value at $memory\left[address\right]$ on $memory$ clock pulse
        \State $memory.buffer \gets memory\left[address\right]$
        \\
        \Statex Write result of read to $dbus$ on $dbus$ clock pulse
        \State $dbus \gets memory.buffer$
    \end{algorithmic}
\end{algorithm}

\noindent Alternately, \cref{alg:lua:memory-access:write} showcases the events triggered on writing a value to a location in memory such as \luainline{proc.memory[0x04] = proc.B}.

\begin{algorithm}[h!]
    \caption{Memory write compilation}
    \label{alg:lua:memory-access:write}
    \begin{algorithmic}[1]
        \Statex \textbf{Input}: $memory$ = Memory unit to read from or write to
        \Statex \textbf{Input}: $address$ = address within $memory$
        \Statex \textbf{Input}: $value$ = Value to write to $memory$
        
        \Statex \textbf{Input}: $abus$ = Bus for address path
        \Statex \textbf{Input}: $dbus$ = Bus for data path
        \Statex \textbf{Side-effect}: $abus$ contains $address$
        \Statex \textbf{Side-effect}: $dbus$ contains $value$ 
        \Statex \textbf{Side-effect}: $memory\left[address\right]$ contains $value$
        \Statex% blank
        
        \Statex Write the $address$ to $abus$
        \State $abus \gets address$
        \Statex Write $value$ to $dbus$
        \State $dbus \gets value$
        \Statex Set $memory.readWrite$ to $WRITE$
        \State $memory.readWrite \gets WRITE$
        \\
        \Statex $memory.buffer$ register gets the $value$ from $dbus$ on the $dbus$ clock pulse
        \State $memory.buffer \gets dbus$
        \\
        \Statex $memory$ writes $value$ to $address$ on $memory$ clock pulse
        \State $memory\left[address\right] \gets memory.buffer$
    \end{algorithmic}
\end{algorithm}

\noindent Once the memory access completes, the bus mechanics reuse the semantics outlined in the previous Assignment \cref{sec:lua:sec:instructions:sec:exec:sec:Assignment}.
    
\subsubsection*{Arithmetic Operations}

Arithmetic operations, e.g. \luainline{operand.B - proc.R}, happen through a single ALU within procsim's model. In the context of \cref{lst:lua:urisc:exec} line 65, when performing the subtraction ALU conditional bit flags based on the \hcmodel{} condition code flags are assigned by default to the \luainline{proc:C}, \luainline{proc:N}, and \luainline{proc:Z}. For multiplication or division operations, \luainline{proc:O} is also set. In the case of URISC, a manual ``buffer'' register of \luainline{proc:R} is utilized as the ``right-hand side'' of it's adder. In the ALU for procsim, both left and right side of a binary operation have ``buffer'' inputs that can be assigned. \Cref{alg:lua:arithmetic} shows the compilation steps for any ALU-based operation. The arithmetic application reuses the previously defined assignment semantics and memory access where required.

\begin{algorithm}[h!]
    \caption{Arithmetic operation compilation}
    \label{alg:lua:arithmetic}
    \begin{algorithmic}[1]
        \Statex \textbf{Input}: $lhs$ = Left-hand side argument
        \Statex \textbf{Input}: $rhs$ = Right-hand side argument
        \Statex \textbf{Input}: $operation$ = Arithmetic operation, e.g. $SUBTRACT$
        \Statex \textbf{Input}: $result$ = Result location
        \Statex \textbf{Input}: $dbus$ = Bus for data path
        \Statex \textbf{Input}: $alu$ = ALU module with condition bit flags
        
        \Statex \textbf{Side-effect}: $dbus$ and $result$ contain result of arithmetic operation
        \Statex \textbf{Side-effect}: $alu$ condition flags are set as appropriate
        \Statex% blank
        
        \Statex Using assignment semantics:
        \State $alu.lhs \gets lhs$
        \State $alu.rhs \gets lhs$
        \State $alu.operation \gets operation$
        \\
        \Statex Start operation
        \State $alu.start \gets START$
        \\
        \State Spin until operation completes, signalled via $alu.done$
        \State $alu$ condition bits are set to appropriate values per the operation
        \State $result \gets alu.result$
    \end{algorithmic}
\end{algorithm}

%\begin{spacing}{1.5}
%    \begin{enumerate}
%        \item Assign \luainline{proc.R} to the right-hand-side ``buffer'' of the ALU
%        \item Assign \luainline{operand.B} to the left-hand-side ``buffer'' of the ALU
%        \item Assign the ALU operation selection signal to ``subtract'' (add after two's complement of right-hand-side)
%        \item If and only if the value is assigned to an lvalue,
%        \begin{enumerate}
%            \item Assign the result of the subtraction result to the data bus
%            \item Store the data bus line value to the location of the lvalue
%        \end{enumerate}
%    \end{enumerate}
%\end{spacing}

\section{Project design and technical flaws}

\subsection{Compiler strain}

While utilizing Lua gave a massive boost in power, speed and functionality of runtime configuration over the previous hc12sim project and procsim.scala, it has large costs to developer productivity. With the amount of software bugs found in the sol library and time spent fixing them, it was a slow process to implement features within the library. Given that sol utilizes high-level variadic template functions and SFINAE compile-time calculations and method resolution techniques \cite{cppreference:SFINAE} to implement most of its features, sol causes most compilers to begin to exponentially increase the time taken to compile \cite{GitHub:ThePhD:sol2:issue:126, GitHub:ThePhD:sol2:issue:295, GitHub:ThePhD:sol2:compilation}. sol's inherent compilation growth caused compilation times for procsim to grow from under 10 minute for a clean build to over 40 minutes as more Lua wrapped functionality was added. More over, the use of templated and SFINAE mechanics caused compiler errors that were near impossible to decipher due to the raw amount of template instantiations present making small errors take massive amounts of time to decipher often resulting in a ``try it and see'' approach to fixing compiler errors. The work completed in \cref{sec:cross-platform:sec:cmake-flows} was started with the goal to try and appease these compiler constraints as reductions in objects to compile would reduce the over-all compiling times -- linking times were inconsequentially affected by sol's combinatorial explosion. In addition to changing how the software was built, as little information was placed into headers as possible when declaring Lua wrappers. This approach allowed for hiding implementations and forcing the compilers to reuse template instantiations as best as possible. However, the times were still unable to lowered below approximately 20 minutes for a clean build. ``ThePhD'' has since created a feature known as \cxxinline{simple_usertype<T>} which provides a runtime-binding mechanism that severely reduces code size and memory contention in most compilers \cite{GitHub:ThePhD:sol2:simple-usertype}. At the time, the \cxxinline{simple_usertype<T>} had significant performance penalties at runtime that have since been removed. The performance penalties of \cxxinline{simple_usertype<T>} were severe enough to impede use within the procsim library.

\subsection{Loss of pedagogical gains for high- versus low-level constructs}

As the features were developed, it became obvious that the ``high-level'' software approach produced a significant hiding of underlying hardware design -- an original concern this approach was meant to combat. In revisiting other tools such as CPU Sim \cite{Skrien2001}, we realized that instead of ``lifting the black box,'' the Lua configurations only shaved several layers off leaving a largely opaque sheet across the model. The model of side-effect-based compilation is extremely powerful and useful as a productivity tool, but it does not benefit students as it hides a lot of wiring details from students. The side-effect based approach to state machine definition provided a higher level approach than projects like \cite{Black2013} but also hid too much of the implementation details from students reducing their hands-on design experience. This ``high-level'' synthesis of a control unit causes students to ignore design considerations surrounding control signals or control unit algorithms. This is particularly evidenced by comparing the control unit for the URISC ISA in \cref{fig:mavaddat1988-urisc-control-unit} to the instruction definition from \cref{lst:lua:urisc-example-2} coupled with the implicit fetch-decode cycle for execution. While the control unit displayed by \cref{fig:mavaddat1988-urisc-control-unit} is not inherently complicated, but it provides a significant design challenge for students to implement it themselves. The procsim configuration does not have any of these signals included and implicitly provides most of them. We believed this to be a direct counter to the pedagogical \cref{req:pedagogical}. The ease of use and agility provided by procsim's configurations is very powerful but costs students learning opportunities. We believed there to be a dissection point between the Lua synthesis of control unit logic and \cite{Black2013}'s approach using a state table. Students require the ability to get their ``hands dirty'' and unfortunately procsim's configurations could not provide enough dirt \cite{Skrien2001,Garcia2009,Ackovska2014,Black2013}.

\begin{figure}[tp!]
    \centering
    \includegraphics[width=0.8\linewidth]{img/mavaddat1988-urisc-control-unit}
    \caption{Control unit for the 16-bit URISC architecture from \cite[p.~331]{Mavaddat1988}.}
    \label{fig:mavaddat1988-urisc-control-unit}
\end{figure}

\section{Analysis of Requirements}

procsim provided a powerful simulation framework design and a lot of extremely strong ideas on how to develop a proper cross-platform, high-performance solution to education of embedded systems and computer architectures. procsim  was proven to run on most major operating systems as required by \cref{req:personal} but required the installation of a program putting it a step behind a purely web-based application. The configurations available were not so deep as to provide complete circuit-level simulation like ShelbySim, CPU Sim or Emumaker86, but it intentionally left out mechanics to try and alleviate the overwhelming nature of a full-blown circuit simulation. Thus procsim met most of the configuration requirements for \cref{req:configuration}. In trading off configuration for pedagogy, procsim provided a solid base for improving students outcomes. But due to the level of information implicitly provided by the side-effect based execution, or outright hidden, it encroached and passed the requirement into becoming detrimental to pedagogy compared to competing projects (\cref{req:pedagogical}). 

Simulations within procsim were fast and provided hooks for all information within components. It did not however provide support for reading signals within the system or investigation of microcode execution. It had support for event-based execution implying different granularities of debugging could be implemented on top of the simulation engine. These features provide most of the requirements for \cref{req:simulations} but procsim had no capabilities for peripheral support at the time. The event-based execution engine would have allowed for granular debugging. Lastly, the IDE provided by the original hc12sim project was more modern than most projects surveyed, but it did not have the features required to keep up with major IDEs. It was our intention to integrate procsim into Eclipse and utilize the Lua Development Toolkit to provide the procsim simulator as an Eclipse-based IDE \cite{Eclipse:LDT}. This would make procsim extremely modern as a fully-featured IDE thanks to Eclipse's plugin architecture. Additionally, procsim utilized cutting edge technologies through Lua and C++11 / C++14 to produce powerful and easy to use software familiar to students. This modern tool set combined with a current IDE would create a modern feeling experience for students alleviating concerns of ``dated'' technologies for \cref{req:modern}.

\begin{table}[h!]
    \centering
    \begin{tabular}{l|cccccc}
        \textbf{Requirements} & \textbf{\hyperref[req:personal]{R1}} & \textbf{\hyperref[req:configuration]{R2}} & \textbf{\hyperref[req:pedagogical]{R3}} & \textbf{\hyperref[req:simulations]{R4}} & \textbf{\hyperref[req:modern]{R5}} & \textbf{Total} \\ \hline
        \textbf{\hyperref[ch:lua]{procsim}} & 
        4 & 3 & 4 & 4 & 4 & \textbf{19} \\
    \end{tabular}
    \caption{Summary of requirement matching for procsim.}
\end{table}
\chapter{Conclusion}
\label{ch:conclusion}

This thesis examined the topic of education for Computer and Software Engineering students on the subject of embedded systems and Instruction Set Architectures. While ultimately unsuccessful in developing a working solution, we have provided a series of known working and invalid approaches to solving the problem outlined in \cref{sec:problem-statement}. We proposed a framework of requirements for a solution to work in the modern laboratory and investigated two major solutions. The first implementation used Scala and the distributed framework Akka to develop a massively asynchronous solution to representing an ISA. In addition, this project investigated the use of a Scala program as a configuration definition. We then implemented a set of novel test and build infrastructure components to better build cross-platform C++ applications. Lastly, we used the knowledge gained from the first and second projects to iterate on an existing project and create a Lua-based configuration system for the desktop. The last chapter included a ``compiler''-like implementation of state-machine representations of a microinstruction and showcased a configuration language capable of defining a full Turing-complete machine. In this chapter, we will summarize the main conclusions from each chapter and use the lessons learned to propose future goals for this research question. 

\section{Scala-based event-driven processor simulator}

\Cref{ch:scala-akka} proposed a web application known as procsim.scala built on the Scala.js platform using Akka and Akka.js to simulate a custom, massively parallel processor in a synchronous environment. We outlined a DSL language for specifying a microinstruction's execution state through in a full programming language environment.

From the procsim.scala project, we proposed several novel concepts for further investigation. First, we utilized Scala to generate a DSL language that was easier for students familiar with 

\section{Contributions}

\todo{Fill this out.. :)}

This thesis is presented in an additional \numberstringnum{\numexpr\totvalue{chapter} - 1\relax} chapters. This thesis discusses technical topics regarding: existing and future pedagogical simulation technologies; development of web applications on modern platforms; building cross-platform native applications; and integration of scripting languages as configuration engines within a native application. We summarize the contributions to each in the following sections.

\subsection{Survey of other and previous works}

Within this section, we provide a quantitative survey of existing simulator technologies in use today analysed against the requirements outlined in \cref{sec:problem-statement}.

\subsection{procsim.scala: a Scala-based event-driven processor simulator for the modern web}

We investigate utilizing Scala and it's compiler back-end Scala.js to build a massively parallel processor simulator for the modern web. Within this investigation, we showcase the following contributions: 

\begin{itemize}
    \item A design for an asynchronous actor-based model for simulation of a custom processor architecture using distributed computing framework Akka
    \item A runtime, compiled VHDL-like DSL for specification of instructions within a custom ISA on top of Scala in the Java Virtual Machine and Scala.js
\end{itemize}

\subsection{Developing cross-platform C++ applications}

We discuss an anecdotal account of refactoring the hc12sim project into a modern C++ environment. We discuss topics involving: developing a multi-platform project and providing adequate quality assurance, improving CMake through custom build scripts, and improving application build times through optimization of build artefact selection. The contributions outlined within this section are: 

\begin{itemize}
    \item An automated platform-specific testing infrastructure using Virtual Machines provisioned with Vagrant and Oracle VirtualBox automated with Jenkins' Pipeline architecture
    \item A collection of project-based CMake scripts to isolate build targets while improving build times, reducing manual configuration and modularizing test specification
\end{itemize}

\subsection{Lua-based configuration-driven processor simulation}

We discuss the use of scripting engines in a native C++ application and their integration points; utilizing Lua as a scripting language within an application; implementation of an application while developing integration tools in parallel; and the pedagogical gains of a high-level language utilized for teaching low-level concepts. We provide the following contributions:

\begin{itemize}
    \item A design for a runtime configuration specification for custom processor architectures through Lua scripts
    \item A state-machine-like representation of a processor instruction execution specified through ``compiling'' a Lua function to microinstruction events
\end{itemize}


\section{Future Work}
\label{sec:future-work}

\begin{enumerate}
    \item Incorporating data collection from the event-loop for processing in software like modelsim
\end{enumerate}

\section{Recommendations}

\subsubsection{Design considerations\cite{Nakamura2013}}

Utilizes a simple dual-BUS architecture for address and data. Uses memory-mapped ``I/O space'' controller for a port-mapped I/O scheme. An interrupt controller is built consisting of a single register, \verb|intr| that stores information about what device interrupted. The authors used a simplified interrupt model in which only one interrupt is supported at a given time. The addition of the interrupt support required the addition of ``RETURN'' and ``CALL'' instructions to support subroutines. Any interrupt implementation must have these machine instructions specified to give the controller the ability to change execution flow whilst maintaining state of the machine. 

\section{Next steps}

\begin{enumerate}
    \item high-level is awesome
    \item Lua is fucking awesome
    \item Sol is fucking awesome
    \item Sol makes compilers sad
    \item Being too fancy with high-level languages creates issues when teaching low-level concepts that leads to ambiguities in teaching ideas
    \item C++ development is insanely slow compared to non-machine languages
\end{enumerate}

%% This adds a line for the Bibliography in the Table of Contents.
\phantomsection
\addcontentsline{toc}{chapter}{Bibliography}
%% ***   Set the bibliography style.   ***

\begin{singlespace}
\printbibliography
\end{singlespace}

%% ***   NOTE   ***
%% If you don't use bibliography files, comment out the previous line
%% and use \begin{thebibliography}...\end{thebibliography}.  (In that
%% case, you should probably put the bibliography in a separate file
%% and \include or \input it here).

%Appendices.
\begin{appendices}
\appendix

\chapter{Procsim headers}
\label{ch:procsim-sources}
\myappendices{\cref{ch:procsim-sources} Procsim headers}

\newcommand{\procsimListing}[1]{%
    \begin{spacing}{1}%
        \captionof{listing}{Source for \texttt{#1}.}%
        \inputminted[breaklines=true]{c++}{./listings/#1}%
    \end{spacing}
    \clearpage%
}

The following listings contain headers describing the core model that is assembled from a Lua configuration such as \cref{lst:lua:urisc-example-1,lst:lua:urisc-example-2}. These headers are supplied to showcase the state produced from Lua configurations. Implementation files are not provided. We have only included the simulation entities for brevity as outlined in \cref{sec:lua:sec:configuration-vs-simulation}.

\section{Memory Components}

\procsimListing{procsim/memory/MemoryUnit.hpp}
\procsimListing{procsim/memory/Register.hpp}
    
\section{Time}

\procsimListing{procsim/time/Clock.hpp}
\procsimListing{procsim/time/AsyncTimerReceiver.hpp}
\procsimListing{procsim/time/Timer.hpp}

\section{Encoding}

\procsimListing{procsim/encoding/Algorithm.hpp}
\procsimListing{procsim/encoding/Code.hpp}
\procsimListing{procsim/encoding/Operand.hpp}

\section{Core}

\procsimListing{procsim/core/Instruction.hpp}
\procsimListing{procsim/core/Proc.hpp}

%\chapter{Course Sentiment Analysis}

% Include information about how it happened

\begin{figure}
    \centering
    \begin{subfigure}{.8\linewidth}
        \centering
        \includegraphics[width=.8\linewidth]{img/course-sentiment-ece-3375}
        \caption{Overall Course Sentiment}
        \label{fig:ece-3375-course-sentiment}
    \end{subfigure}
    
    \begin{subfigure}{.8\linewidth}
        \centering
        \includegraphics[width=.8\linewidth]{img/lab-sentiment-ece-3375}
        \caption{Lab Sentiment}
        \label{fig:ece-3375-lab-sentiment}
        \todofig{Fix chart formatting, names}
    \end{subfigure}
    
    \caption{ECE 3375 - Course Sentiment Analysis\cite{evals:ece3375-2013, evals:ece3375-2014}}
\end{figure}
\end{appendices}

%CV only relevant stuff... not full CV.
\phantomsection
\addcontentsline{toc}{chapter}{Curriculum Vitae}
\chapter*{Curriculum Vitae}
\begin{singlespace}
\begin{table}[!ht]
\begin{tabular}{ll}
\textbf{Name:} & \firstname{} \lastname\\\\
\textbf{Post-Secondary} & The University of Western Ontario\\
\textbf{Education and}& London, ON\\
\textbf{Degrees:} & \begin{tabular}{lll}
    2009 - 2015 & BESc. & Computer Engineering \\
                & BSc. & Computer Science \\
                \end{tabular} \\
& The University of Western Ontario\\
& London, ON\\
& 2015 - 2017 MESc.\\\\
\textbf{Honours and}& Ontario Graduate Scholarship\\
\textbf{Awards:}& 2015-2016\\\\
& Best Presentation - Software Engineering \\
& Electrical and Computer Engineering Graduate Symposium \\
& Spring 2016 \\\\
\textbf{Related Work}& Limited Duties Instructor \\
\textbf{Experience:}& Data Structures and Algorithms for Software Engineering\\
& The University of Western Ontario\\
& Winter 2017\\\\
& Teaching Assistant\\
& The University of Western Ontario\\
& 2015 - 2016\\
\end{tabular}
\end{table}
\end{singlespace}
%\subsubsection*{Publications:}
%:(
%La La
\end{document}

