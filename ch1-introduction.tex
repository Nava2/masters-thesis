\chapter{Introduction}

\todo{Educational justification}
\todo{comparison of other simulators}

\begin{comment}
\begin{itemize}

\item What research question(s) are you asking?
\begin{itemize}
\item Can improvements to software simulation stories improve learning outcomes for students in electrical, computer and software engineering on topics surrounding microcontrollers? 
\item Do familiar high-level software constructs improve student learning outcomes of low-level hardware components?
\item What are the requirements needed for a simulation suite to be successful in a pedagogical space rather than industrial?
\item What curriculum changes are required to improve student outcomes?
\end{itemize}

\item Why are you asking it/them?
\begin{itemize}
\item Drop in marks for students within these courses
\item Students voice frustration with course material and current solutions
\end{itemize}

\item Has anyone else done anything similar?
\begin{itemize}
\item Yes. 
\end{itemize}
\item Is your research relevant to research/practice/theory in your field?
\begin{itemize}
\item Education is always required
\item Improvements are required as IoT increases and devices are moving back towards small
microelectronics
\end{itemize}
\item What is already known or understood about this topic?
\begin{itemize}
\item Simulations: \cite{Tappan2009}, \cite{Skrien2001}, \cite{Skillen2011}
\item Curriculum: 
\end{itemize}
\item How might your research add to this understanding, or challenge existing theories and beliefs?
\begin{itemize}
\item Extend on existing solution
\item Propose opportunity to replace multiple configurations, not a single one
\end{itemize}
\end{itemize}
\end{comment}

Embedded systems and computer architectures are a critical part of both computer and software engineering undergraduate programs\cite[appendix, p.~73-76]{cec2016}\cite[p.~33]{sec2015}\cite{ece-ce-program, Ristov2011, Stolikj2011}. Over time, it is expected that knowledge of embedded systems and computer architectures will be required given the rapid growth of positions in both systems software developers (+13\%) and computer occupations (+12.5\%)\cite{bls2014}. At the author's institution, The University of Western Ontario, two courses cover embedded systems and computer architectures, ECE 3375 - Microprocessors and Microcomputers and ECE 3390B - Hardware/Software Co-Design\cite{eceOutlines}. These topics are taught through the use of industry based software such as Intel Quartus Prime (formerly Altera Quartus II)\cite{quartus} or Xilinx ISE WebPACK\cite{xilinxISE}.

\section{Research Question}

This thesis attempts to answer the following research question:

\begin{displayquote}
Can improvements and increased usage of software simulation technologies within laboratory exercises in undergraduate embedded system and computer architecture courses improve student engagement in exercises?
\end{displayquote}

Given the overall research question, there leads several sub-questions: 

\begin{enumerate}
	\item Are simulation elements considered beneficial to student outcomes?
	\item What elements of simulation software are required for successful implementation within an undergraduate course?
\end{enumerate}

Each of these questions are elaborated in further sections. 

\section{Motivation}

Students in computer engineering have a growing requirement to have strong knowledge of both embedded systems and computer architectures due to increases in computer engineering careers\cite[p.~30]{cec2016}\cite{bls2014}. Unfortunately, many students do not show enthusiasm for these subjects. 

At Western University, students in all programs in Electrical \& Computer Engineering must take the course ECE 3375 - Microprocessors and Microcomputers\cite{uwo-we-programprogression, eceoutline-ece3375}. For example, shown in \cref{table-course-enthusiasm-ece3375}\todo{Update with 2015/2016 data if provided}, for a required course in embedded systems at Western University (ECE 3375)\cite{eceOutlines}, student enthusiasm for the course is extremely mixed. Unfortunately, there does not exist data broken down between computer and software engineering -- both are required to take this course. Additionally, with primitive sentiment analysis applied to \cite{evals:ece3375-2013, evals:ece3375-2014}, shown in \cref{fig:ece-3375-course-sentiment}, we can see that after completion of the course, students are overall happy with the course content in itself as 72.73\% of students found the experience positive. However, by extracting comments regarding the laboratory sessions and applying the same technique for sentiment as the course, shown in \cref{fig:ece-3375-lab-sentiment} \todo{Fix chart formatting, names}, we find that students are likely very unhappy with the lab components as 33\% stated negative reviews of the component. 

\begin{table}[!h]
    \centering
    \begin{tabular}{lc|r|r|r}
        \multicolumn{5}{c}{Initial Level of Enthusiasm} \\ \hline\hline
        \multicolumn{2}{l}{Year} & \multicolumn{1}{|c}{2013} & \multicolumn{1}{|c}{2014} & \multicolumn{1}{|c}{Total} \\ \hline
        High          & \%       & 45.6                     & 57.8                     & 53.0                      \\ 
        Medium        & \%       & 54.2                     & 33.3                     & 41.6                      \\
        Low           & \%       & 0.0                      & 8.9                      & 5.4                       \\ 
        Total         & \#       & 59                       & 90                       & 149                       \\ \hline
    \end{tabular}
    \caption{ECE 3375 Course ``Level of enthusiasm before taking the course'' as reported by undergraduate students\cite{evals:ece3375-2013, evals:ece3375-2014}} 
    \label{table-course-enthusiasm-ece3375}
\end{table}

\begin{figure}
    \centering
    \begin{subfigure}{.8\linewidth}
       \centering
       \includegraphics[width=.8\linewidth]{img/course-sentiment-ece-3375}
       \caption{Overall Course Sentiment}
       \label{fig:ece-3375-course-sentiment}
    \end{subfigure}
   
    \begin{subfigure}{.8\linewidth}
       \centering
       \includegraphics[width=.8\linewidth]{img/lab-sentiment-ece-3375}
       \caption{Lab Sentiment}
       \label{fig:ece-3375-lab-sentiment}
    \end{subfigure}

    \caption{ECE 3375 - Course Sentiment Analysis\cite{evals:ece3375-2013, evals:ece3375-2014}}
\end{figure}

Researchers have found that there are concerns with low-level hardware constructs being taught to high-level Computer Science, Software Engineering and Computer engineering graduates \cite{Ristov2011, Stolikj2011}. Additionally, the author can corroborate the following claims by \cite{Ristov2011}: 
\begin{displayquote}
    \begin{enumerate}
        \item Disjointed material between lectures, theoretical and practical
        exercises. Students could not transfer knowledge gained from either lectures or theoretical exercises to practical exercises [\dots] 
            \label{quote:ristov:1}
        \item Inappropriate programming and simulation environment [\dots] 
            \label{quote:ristov:2}
        \item No ``real world'' application [\dots] 
            \label{quote:ristov:3}
    \end{enumerate}
\end{displayquote}
Given the author's first-hand experiences in both taking and assisting in the instruction of ECE3375, these claims are repeated by students during laboratory exercises and review sessions creating hostility towards the subject matter. In a direct comparison to \cref{quote:ristov:2}, students in ECE3375 use the WinIDE12Z Integrated Development Environment\cite{winide}, shown in \cref{fig:winide-screenshot-pemicro}. In ECE3375, students found WinIDE complicated, frustrating and ``outdated'' \cite{evals:ece3375-2013, evals:ece3375-2014} both \cref{quote:ristov:2} and \cref{quote:ristov:3} of \cite{Ristov2011}'s observations. Additionally students found it difficult to move from lecture material that presented embedded systems as ``translucent'' devices composed of inner components to the literal ``black box'' of the physical hardware -- shown in \cref{fig:hc12-board}. 

\begin{figure}[!ht]
    \centering
    \includegraphics[width=0.7\linewidth]{img/winide-screenshot-pemicro}
    \caption{WinIDE12Z Integrated Development Environment\cite{winide-screenshot}}
    \label{fig:winide-screenshot-pemicro}
\end{figure}

\begin{figure}[!hb]
    \centering
    \includegraphics[width=0.8\linewidth]{img/board-on-cropped}
    \caption{Freescale S12CPUV2 development board built by Western Engineering Electronics Shop for use in ECE 3375}
    \label{fig:hc12-board}
\end{figure}

Given sentiment analysis and the comments provided by students, the author extracted several propositions for improving laboratory sessions: 

\begin{enumerate}%[label={\textbf{\theproposition \arabic*}}]
    \item Increase the number of laboratory sessions 
        \label[proposition]{prop:add-lab-sessions}
    \item Add tutorial sessions to course schedule
        \label[proposition]{prop:add-tutorial-sessions}
    \item Change course material to use modern processors
        \label[proposition]{prop:change-course-materials}
    \item Replace laboratory microcontrollers with different hardware
        \label[proposition]{prop:replace-hardware} % REPEAL AND REPLACE!11!!11
    \item Improve and/or replace software within laboratory
        \label[proposition]{prop:replace-software}
\end{enumerate}

All of these considerations were bought to the course instructor and the following limitations were noted for each. For \cref{prop:add-lab-sessions,,prop:add-tutorial-sessions}, while the addition of more hands-on experience has been shown to improve student outcomes\cite{Ristov2011, Stolikj2011} due to the large amount of financial resources required to add more laboratory sessions, this option is rarely available for many programs. In addition, due to the layout of the current course laboratory sessions, there does not exist enough weeks in the 12-week course to add more sessions given the current resources. Thus, both \cref{prop:add-lab-sessions,,prop:add-tutorial-sessions} can not be implemented given financial and temporal restrictions.

\Cref{prop:change-course-materials} asks the instructor to change the current course material to use more modern hardware/software. There are several reasons why this is not beneficial to the improvement of learning outcomes. The current architecture used is the Freescale S68HC12. This architecture provides a simple interface with only six 16-bit addressable registers (\verb|A|, \verb|B|, \verb|SP|, \verb|PC|, \verb|X|, \verb|Y|) and multiple addressing modes\cite{hc12Manual2006}. The software abstractions of the S68HC12 is similar to current Intel\textregistered{} x86 processors\cite{intel2017}. The parallelism between architectures allows instructors to use more simple architectures that still allow students to apply concepts to higher-level architectures while simultaneously reducing the cognitive ``base knowledge'' required to succeed. Using simpler architectures like the Ultimate Reduced Instruction Set Computer (URISC, \cite{mavaddat1988}) has been shown to be very applicable to improving student outcomes\cite{Nakamura2013, McLoughlin2010, mavaddat1988}. Additionally, by changing course materials, it leads into changing the laboratory hardware which \cref{prop:replace-hardware} discusses. Unfortunately, the replacing laboratory equipment is requires significant investment and with justifications for \cref{prop:change-course-materials} showcasing that there is not noticeable improvements in learning outcomes, the cost-benefit analysis does not prove worthwhile to adjust both \cref{prop:change-course-materials,,prop:replace-hardware}.

The last proposition, \cref{prop:replace-software} is the most attainable proposition. Researchers have found that improving software technology in the classroom and laboratory can vastly improve the learning outcomes of students while improving engagement \cite{Ackovska2014, Stolikj2011, Ristov2011, Ristov2014, Nikolic2009, Skillen2011, Tappan2009, Djordjevic2005, cec2016}. Unfortunately, within ECE3375 students are immediately presented with the interfaces shown in \cref{fig:winide-screenshot-pemicro}. Students have expressed immense frustrations with the software when used in the laboratory exercises and often encounter hardware issues that are not caused by their own work\cite{evals:ece3375-2013, evals:ece3375-2014}. For other hardware solutions, \textquote[\cite{Ackovska2014}]{students were faced with hardware-specific problems such as sensor failure or unconnected pins.} These errors are hard to debug, if not impossible for students with no prior knowledge. Teaching assistants and instructors are often unable to debug these issues without the aid of technicians. These hard-to-diagnose hardware issues lead to strained teaching resources and frustrations. Thus, creating a large gain by replacing this ageing software with new tooling. Thus, this thesis looks to investigate the requirements surrounding replacing the current software in use within the ECE3375 course and propose a solution for others to utilize in other courses. 

\subsection{Evaluation of current simulation technologies} 



\subsubsection{ShelbySim}

\cite{Tappan2009}

ShelbySim is an education-oriented software system for designing, simulating, and evaluating computer-engineering based applications. ShelbySim was designed surrounding a new Java-like programming language including a compiler explicitly built around providing extensive diagnostic information such as logging, tracing, and inspection capabilities. These tools provide students with raw data for quantitative analysis, evaluation and reporting of their designs. 

The software is open-source, though not available, and is written using Java allowing full operating system independent support. Additionally, 3D visualized results are provided for viewing developed components. ShelbySim is broken down into three subcomponents:
\begin{enumerate}
\item Software component - a custom programming language (Shelby), a compiler, and an interpretation runtime
\item Hardware component - filling a similar niche to MultiSim, but with tight integration with Shelby and its underlying tracing. Additional support exists for external component integration
\item Simulation component - providing a deterministic and stochastic approach for inputs into custom hardware versions
\end{enumerate} 

Provides evaluation criterion for students components and underlying systems. The simulated components have parameters that are modifiable through switches and sliders (e.g. \{on, off\} or a range from 0 - 100\%). This gives students metrics to evaluate their designs. Additionally, outputs are exported at runtime to Comma Separated Value (CSV) files allowing for more in-depth analysis with external programs such as Microsoft Excel or MATLAB. This gives a flexible and realistic testing environment for student learning. 

\subsection{Talking Points}

\begin{itemize}
\item Most software does not focus on pedagogy
\item Industry software hides underlying information (rightly so for them)
\item Visualization gives "less opaque" view of the system
\item Language 
\item Focuses on compiler semantics, low-level detail implementations (e.g. motor characteristics)
\end{itemize}

\section{CPUSim}

\cite{Skrien2001}

CPU Sim is a Java CPU simulator written for use within a classroom environment. CPUSim allows students to design, modify, and compare various computer architectures at the register-transfer level and higher. Additionally, students may write and debug assembly code for custom architectures. The JavaFX front-end allows for viewing CPU internal components (e.g. RAMs and Registers) while stepping through programs. CPUSim allows students to encode and decode values through a user interface for machine codes. 

CPUSim also allows students to specify microinstructions that are combined to create full assembly-level instructions. This forces students to describe and think about what actions they need for their ``higher-level'' instructions. 

CPUSim's design is flexible given the Java-based system allowing students to work on multiple different platforms easily. Additionally, it is written such that many aspect of the software may be customized. Unfortunately, due to the decision to utilize the Java Virtual Machine, interfaces with lower-level components such as serial ports is difficult. 

\subsection{Talking Points}

\begin{itemize}
\item Java-based
\item Full-debugger
\item Specify microinstructions (microcodes) instead of assuming them
\item Code is older, tightly coupled and problematic
\item 
\end{itemize}

\section{EASE}

\cite{Skillen2011}

\blockquote[Skillen2011]{Striking the right balance between teaching sufficient de-tails of hardware components and their working principles, and important theoretical concepts useful for programming the computer is always a challenge.}

Found most of the simulators in \cite{Nikolic2009} were inadequate for teaching simulation architecture courses, driving the work in EASE. Many were used for circuitry/RTL level work thus not good enough for project at hand. 

The following characteristics are most important: 

\begin{enumerate}
\item Support of more than one architecture to illustrate CISC, RISC and URISC
\item Designed in a modular way to allow for extension of the simulation (adding new instructions)
\item Source code available
\item Portable
\end{enumerate}

All of which are now (or will be) fixed in future versions of CPUSim. The article compared approximately version 3.1 (\cite{Skrien2001}). 

Requires the following further improvements: 

\begin{itemize}
\item Add built-in editor
\item No provided documentation or samples
\item 
\end{itemize}

\subsection{EASE vs. CPUSim}

Architectures are specified outside the simulation itself as a Java Library -- disconnected from the simulator/software itself. 

Architectures are bound to specific models that are specified by EASE, (e.g. CISC (similar to x86), RISC and a URISC). These are tightly coupled to EASE itself. 

Based on descriptions of the internal mechanics, it is tightly coupled to the current structure and not flexible in changing internals -- though, the source is not available. 


\subsection{Talking Points}

\begin{itemize}
\item Java-based
\item CPUSim is directly compared in this paper
\item No source available (yet claims it is..)
\item Dead-ware
\item Tightly coupled to specified architectures
\item Focuses on assembly in these architectures
\item Ships with URISC architecture (subleq)
\end{itemize}

\section{Survey/eval of simulators for teaching}

\cite{Nikolic2009}

Mixture of discussion surrounding architectures and organization -- less on simulation of actual assembly-level programming.  


\subsection{Talking Points}

\begin{itemize}
\item Java-based
\item CPUSim is directly compared in this paper
\item No source available (yet claims it is..)
\item Dead-ware
\item Tightly coupled to specified architectures
\item Focuses on assembly in these architectures
\item Ships with URISC architecture (subleq)
\end{itemize}

\section{A Perspective on the Experiential Learning of Computer Architecture}

\cite{McLoughlin2010, Nakamura2013}

Discussion of an MSc. curriculum based around designing a CPU called ``TinyCPU'' \cite{McLoughlin2010} and an extension ``TinyCSE'' \cite{Nakamura2013}. The MSc. program detailed is very similar to Western's computer engineering undergraduate program in content/topics. 

\subsection{Design considerations}

Utilizes a simple dual-BUS architecture for address and data. Uses memory-mapped ``I/O space'' controller for a port-mapped I/O scheme. An interrupt controller is built consisting of a single register, \verb|intr| that stores information about what device interrupted. The authors used a simplified interrupt model in which only one interrupt is supported at a given time. The addition of the interrupt support required the addition of ``RETURN'' and ``CALL'' instructions to support subroutines. Any interrupt implementation must have these machine instructions specified to give the controller the ability to change execution flow whilst maintaining state of the machine. 


\subsection{Talking Points}

\begin{itemize}
\item Verilog HDL-based
\item Requires use of complex suites like Modelsim or Altera
\item To be used directly on FPGAs
\item Comes with a C compiler/assembler
\item Proved implementation of multiple device controllers
\item Bound to a single architecture
\end{itemize}

\section{Computer Engineering Curricula 2016}

\cite{cec2016}

TODO Find the information surrounding CE topics and how they relate to our project. 

\subsection{Talking Points}

\begin{itemize}
\item Verilog HDL-based
\item Requires use of complex suites like Modelsim or Altera
\item To be used directly on FPGAs
\item Comes with a C compiler/assembler
\item Proved implementation of multiple device controllers
\item Bound to a single architecture
\end{itemize}

%\begin{figure}[ht]
%\begin{center}
%\includegraphics[height = 9cm, width = 9cm]{pic1.jpeg}
%\caption{A long memory time series\label{ts1}}
%\end{center}
%\end{figure}

%Here's a table.
%\begin{table}[ht]
%\begin{center}
%\begin{tabular}[ht]{|c|lr|c|} 
%%c stands for centre, l for left, r for right; the | puts lines in between, and the hline puts a horizontal line in
%\hline
%$n$ & $\alpha$ &$n\alpha$ & $\beta$\\
%\hline
%1 & 0.2 & 0.2 & 5\\
%\hline
%2 & 0.3 & 0.6 & 4\\
%\hline
%3 & 0.7 & 2.1 & 3\\
%\hline
%\end{tabular}
%\caption{A random table \label{tab1}}
%\end{center}
%\end{table}
%
%\begin{eqnarray}
%y &=& mx + b \label{eq1}\\
%&=& ax+ c
%\label{eq2}
%\end{eqnarray}
%
%This is an un-numbered equation, along with a numbered one. 
%\begin{eqnarray}
%u &=& px \nonumber\\
%p &=& P(X=x) \label{eqn3}
%\end{eqnarray}
%
%Look at Table \ref{tab1} and Figure \ref{ts1} and equations \ref{eq1},  \ref{eq2}, and \ref{eqn3}.
%
%Let's do some matrix algebra now.
%
%\begin{equation}
%det\left(\left|\begin{array}{ccc} 2 & 3 & 5\\
%4 & 4 & 6\\
%9 & 8 & 1
%\end{array}\right|\right) = 42
%\end{equation}
%
%In the equation and eqnarray environments, you don't need to have the dollar sign to enter math mode.
%
%\begin{eqnarray}
%\alpha = \beta_1 \Gamma^{-1}
%\end{eqnarray}
%
%This is citing a reference ~\cite{mygood11111}.  This is citing another ~\cite{mrx05}.  Nobody said something ~\cite{Nobody06}.
