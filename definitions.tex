
\chapter*{List of Definitions}

\begin{definition}[Embedded Systems] 
	Topics involving characteristics of embedded systems, techniques for embedded applications, parallel input and output, synchronous and asynchronous serial communication, interrupt handling, applications involving data acquisition, control, sensors, and actuators, implementation strategies for complex embedded systems \cite[p.~118]{cec2016}
\end{definition}

\begin{definition}[Computer Architectures] 
	Topics involving computer organization and architecture. Including, but not limited to processor organization, instruction set architecture, memory system organization, performance, and interfacing fundamentals\cite[p.~118]{cec2016}
\end{definition}

\begin{definition}[ISA]
    Instruction Set Architecture. The design of an interface between software and hardware designs in a computing environment. Once realized, it is referred to as an implementation. 
\end{definition}

\begin{definition}[RISC]
    Reduced Instruction Set Computer. A type of microprocessor architecture that utilizes small, highly-optimized set of instructions used to reduce power and improve instruction performance\cite{Aletan1992, Stokes1999}. These architectures are commonly used in embedded, mobile (ARM), and high-performance server applications (IBM POWER). 
\end{definition}

\begin{definition}[CISC]
    Complex Instruction Set Computer. A type of microprocessor architecture that utilizes larger, instructions aiming to reduce the software instructions required to complete a task\cite{Aletan1992, Stokes1999}. CISC architectures are used in most general applications based on Intel\textregistered{} x86 ISA\cite{intel2017}.
\end{definition}

\begin{definition}[IDE]
    Integrated Development Environment. A tool used for developing hardware and software components that provides integrated tools for the current development context. 
\end{definition}