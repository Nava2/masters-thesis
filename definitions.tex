
\chapter*{List of Definitions}

\begin{definition}[Application Programming Interface (API)]
    The layer between a developer that utilizes a library or program and the library or program itself. APIs are designed to hide implementation details and ease developer usage.
\end{definition}

\begin{definition}[Embedded Systems] 
	Topics involving characteristics of embedded systems, techniques for embedded applications, parallel input and output, synchronous and asynchronous serial communication, interrupt handling, applications involving data acquisition, control, sensors, and actuators, implementation strategies for complex embedded systems \cite[p.~118]{cec2016}
\end{definition}

\begin{definition}[Computer Architectures] 
	Topics involving computer organization and architecture. Including, but not limited to processor organization, instruction set architecture, memory system organization, performance, and interfacing fundamentals\cite[p.~118]{cec2016}
\end{definition}

\begin{definition}[Instruction Set Architecture (ISA)]
    The design of an interface between software and hardware designs in a computing environment. Including but not limited to assembly language and microcode design. Once realized, an ISA is referred to as an Instruction Set Implementation. 
\end{definition}

\begin{definition}[Reduced Instruction Set Computer (RISC)]
    A type of microprocessor architecture that utilizes small, highly-optimized set of instructions used to reduce power and improve instruction performance \cite{Aletan1992, Stokes1999}. These architectures are commonly used in embedded, mobile (ARM), and high-performance server applications (IBM POWER). 
\end{definition}

\begin{definition}[Complex Instruction Set Computer (CISC)]
    A type of microprocessor architecture that utilizes larger, instructions aiming to reduce the software instructions required to complete a task \cite{Aletan1992, Stokes1999}. CISC architectures are used in most general applications based on Intel\textregistered{} x86 ISA \cite{intel2017}.
\end{definition}

\begin{definition}[Integrated Development Environment (IDE)]
    A tool used for developing hardware and software components that provides integrated tools for the current development context. For example, Xilinx ISE \cite{xilinxISE}, JetBrains IntelliJ IDEA\footnote{Available: \url{https://www.jetbrains.com/idea/}}.
\end{definition}

\begin{definition}[Hardware Description Language (HDL)]
    A programming language used to describe the structure and behaviour of electronic circuits. These languages are used to synthesize simulated and physical circuit layout from the gate and register transfer level \cite{Chu2006}.
\end{definition}

\begin{definition}[Graphical User Interface (GUI)]
    Any interface based on rendering graphically, for example a window with buttons. 
\end{definition}

\begin{definition}[Arithmetic Logic Unit (ALU)]
    In a computer system, ALUs are responsible for performing arithmetic operations on binary input such as addition, shifts, multiplication, division.
\end{definition}

\begin{definition}[Domain-Specific Language (DSL)]
    A computer programming language that is specialized to a particular application space (domain). For example, HTML is a DSL used to layout websites but could not be used to program an embedded device. Similarly, C++ is an excellent language for creating applications but is very poor at describing layout for a webpage.
\end{definition}

\begin{definition}[Transpiler]
    A compiler that compiles one language to another. Typical transpilers convert one language to JavaScript to allow use in web applications. For example, Scala.js and TypeScript both compile their respective language to JavaScript.
\end{definition}

\begin{definition}[Reification]
    The process by which an abstraction (usually a \javainline{String} representation) of software is converted into an explicit object. For example, JavaScript's \jsinline{eval(script)} compiles and executes the content of the \verb|script| passed and any state changes within the script are applied to the global scope.
\end{definition}

\begin{definition}[Just-in-Time Compiler (JIT)]
    A traditional compiler that operates at runtime to perform optimization on running code. JITs are typically found in interpreted languages such as on the JVM or in JavaScript. JITs vastly improve the runtime of the code they compile as they often utilize semantics such as code path tracing and known constant analysis that is not necessarily available at compile-time. 
\end{definition}