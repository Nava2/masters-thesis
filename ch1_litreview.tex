\chapter{Introduction}

\todo{Educational justification}
\todo{comparison of other simulators}

Embedded systems and computer architectures are a critical part of both computer and software engineering undergraduate programs\cite[appendix, p.~73-76]{cec2016}\cite[p.~33]{sec2015}. Over time, it is expected that knowledge of embedded systems and computer architectures will be required given the rapid growth of positions in both systems software developers (+13\%) and computer occupations (+12.5\%)\cite{bls2014}. At the authors institution, The University of Western Ontario, two courses cover embedded systems and computer architectures, ECE 3375 - Microprocessors and Microcomputers and ECE 3390B - Hardware/Software Co-Design\cite{eceOutlines}. These topics are taught through the use of industry based software such as Intel Quartus Prime (formerly Altera Quartus II)\cite{quartus} or Xilinx ISE WebPACK\cite{xilinxISE}.

\section{Research Question}

This thesis attempts to answer the following research question:

\begin{displayquote}
Can improvements and increased usage of software simulation technologies within laboratory exercises in undergraduate embedded system and computer architecture courses improve student engagement in exercises?
\end{displayquote}

Given the overall research question, there leads several sub-questions: 

\begin{enumerate}
	\item Are simulation elements considered beneficial to student outcomes?
	\item What elements of simulation software are required for successful implementation within an undergraduate course?
\end{enumerate}

Each of these questions are elaborated in further sections. 

\section{Motivation}

Students in both computer engineering and software engineering have a growing requirement to have strong knowledge of both embedded systems


This thesis presents a novel iteration of simulation software to improve the learning outcomes embedded systems for students in computer and software engineering programs. Currently available software packages are often needlessly complicated, expensive and overwhelming for introductory students. 

\begin{itemize}

\item What research question(s) are you asking?
\begin{itemize}
\item Can improvements to software simulation stories improve learning outcomes for students in electrical, computer and software engineering on topics surrounding microcontrollers? 
\item Do familiar high-level software constructs improve student learning outcomes of low-level hardware components?
\item What are the requirements needed for a simulation suite to be successful in a pedagogical space rather than industrial?
\item What curriculum changes are required to improve student outcomes?
\end{itemize}

\item Why are you asking it/them?
\begin{itemize}
\item Drop in marks for students within these courses
\item Students voice frustration with course material and current solutions
\end{itemize}

\item Has anyone else done anything similar?
\begin{itemize}
\item Yes. 
\end{itemize}
\item Is your research relevant to research/practice/theory in your field?
\begin{itemize}
\item Education is always required
\item Improvements are required as IoT increases and devices are moving back towards small
	microelectronics
\end{itemize}
\item What is already known or understood about this topic?
\begin{itemize}
\item Simulations: \cite{Tappan2009}, \cite{Skrien2001}, \cite{Skillen2011}
\item Curriculum: 
\end{itemize}
\item How might your research add to this understanding, or challenge existing theories and beliefs?
\begin{itemize}
\item Extend on existing solution
\item Propose opportunity to replace multiple configurations, not a single one
\end{itemize}
\end{itemize}

\section{ShelbySim \cite{Tappan2009}}

ShelbySim is an education-oriented software system for designing, simulating, and evaluating computer-engineering based applications. ShelbySim was designed surrounding a new Java-like programming language including a compiler explicitly built around providing extensive diagnostic information such as logging, tracing, and inspection capabilities. These tools provide students with raw data for quantitative analysis, evaluation and reporting of their designs. 

The software is open-source, though not available, and is written using Java allowing full operating system independent support. Additionally, 3D visualized results are provided for viewing developed components. ShelbySim is broken down into three subcomponents:
\begin{enumerate}
\item Software component - a custom programming language (Shelby), a compiler, and an interpretation runtime
\item Hardware component - filling a similar niche to MultiSim, but with tight integration with Shelby and its underlying tracing. Additional support exists for external component integration
\item Simulation component - providing a deterministic and stochastic approach for inputs into custom hardware versions
\end{enumerate} 

Provides evaluation criterion for students components and underlying systems. The simulated components have parameters that are modifiable through switches and sliders (e.g. \{on, off\} or a range from 0 - 100\%). This gives students metrics to evaluate their designs. Additionally, outputs are exported at runtime to Comma Separated Value (CSV) files allowing for more in-depth analysis with external programs such as Microsoft Excel or MATLAB. This gives a flexible and realistic testing environment for student learning. 

\subsubsection{Talking Points}

\begin{itemize}
\item Most software does not focus on pedagogy
\item Industry software hides underlying information (rightly so for them)
\item Visualization gives "less opaque" view of the system
\item Language 
\item Focuses on compiler semantics, low-level detail implementations (e.g. motor characteristics)
\end{itemize}

\section{CPUSim \cite{Skrien2001}}

CPU Sim is a Java CPU simulator written for use within a classroom environment. CPUSim allows students to design, modify, and compare various computer architectures at the register-transfer level and higher. Additionally, students may write and debug assembly code for custom architectures. The JavaFX front-end allows for viewing CPU internal components (e.g. RAMs and Registers) while stepping through programs. CPUSim allows students to encode and decode values through a user interface for machine codes. 

CPUSim also allows students to specify microinstructions that are combined to create full assembly-level instructions. This forces students to describe and think about what actions they need for their ``higher-level'' instructions. 

CPUSim's design is flexible given the Java-based system allowing students to work on multiple different platforms easily. Additionally, it is written such that many aspect of the software may be customized. Unfortunately, due to the decision to utilize the Java Virtual Machine, interfaces with lower-level components such as serial ports is difficult. 

\subsubsection{Talking Points}

\begin{itemize}
\item Java-based
\item Full-debugger
\item Specify microinstructions (microcodes) instead of assuming them
\item Code is older, tightly coupled and problematic
\item 
\end{itemize}

\section{EASE \cite{Skillen2011}}

\blockquote[Skillen2011]{Striking the right balance between teaching sufficient de-tails of hardware components and their working principles, and important theoretical concepts useful for programming the computer is always a challenge.}

Found most of the simulators in \cite{Nikolic2009} were inadequate for teaching simulation architecture courses, driving the work in EASE. Many were used for circuitry/RTL level work thus not good enough for project at hand. 

The following characteristics are most important: 

\begin{enumerate}
\item Support of more than one architecture to illustrate CISC, RISC and URISC
\item Designed in a modular way to allow for extension of the simulation (adding new instructions)
\item Source code available
\item Portable
\end{enumerate}

All of which are now (or will be) fixed in future versions of CPUSim. The article compared approximately version 3.1 (\cite{Skrien2001}). 

Requires the following further improvements: 

\begin{itemize}
\item Add built-in editor
\item No provided documentation or samples
\item 
\end{itemize}

\subsection{EASE vs. CPUSim}

Architectures are specified outside the simulation itself as a Java Library -- disconnected from the simulator/software itself. 

Architectures are bound to specific models that are specified by EASE, (e.g. CISC (similar to x86), RISC and a URISC). These are tightly coupled to EASE itself. 

Based on descriptions of the internal mechanics, it is tightly coupled to the current structure and not flexible in changing internals -- though, the source is not available. 


\subsection{Talking Points}

\begin{itemize}
\item Java-based
\item CPUSim is directly compared in this paper
\item No source available (yet claims it is..)
\item Dead-ware
\item Tightly coupled to specified architectures
\item Focuses on assembly in these architectures
\item Ships with URISC architecture (subleq)
\end{itemize}

\section{Survey/eval of simulators for teaching \cite{Nikolic2009}}

Mixture of discussion surrounding architectures and organization -- less on simulation of actual assembly-level programming.  


\subsection{Talking Points}

\begin{itemize}
\item Java-based
\item CPUSim is directly compared in this paper
\item No source available (yet claims it is..)
\item Dead-ware
\item Tightly coupled to specified architectures
\item Focuses on assembly in these architectures
\item Ships with URISC architecture (subleq)
\end{itemize}

\section{A Perspective on the Experiential Learning of Computer Architecture \cite{McLoughlin2010}, \cite{Nakamura2013}}

Discussion of an MSc. curriculum based around designing a CPU called ``TinyCPU'' \cite{McLoughlin2010} and an extension ``TinyCSE'' \cite{Nakamura2013}. The MSc. program detailed is very similar to Western's computer engineering undergraduate program in content/topics. 

\subsection{Design considerations}

Utilizes a simple dual-BUS architecture for address and data. Uses memory-mapped ``I/O space'' controller for a port-mapped I/O scheme. An interrupt controller is built consisting of a single register, \verb|intr| that stores information about what device interrupted. The authors used a simplified interrupt model in which only one interrupt is supported at a given time. The addition of the interrupt support required the addition of ``RETURN'' and ``CALL'' instructions to support subroutines. Any interrupt implementation must have these machine instructions specified to give the controller the ability to change execution flow whilst maintaining state of the machine. 


\subsection{Talking Points}

\begin{itemize}
\item Verilog HDL-based
\item Requires use of complex suites like Modelsim or Altera
\item To be used directly on FPGAs
\item Comes with a C compiler/assembler
\item Proved implementation of multiple device controllers
\item Bound to a single architecture
\end{itemize}

\section{Computer Engineering Curricula 2016 \cite{cec2016}}

TODO Find the information surrounding CE topics and how they relate to our project. 

\subsection{Talking Points}

\begin{itemize}
\item Verilog HDL-based
\item Requires use of complex suites like Modelsim or Altera
\item To be used directly on FPGAs
\item Comes with a C compiler/assembler
\item Proved implementation of multiple device controllers
\item Bound to a single architecture
\end{itemize}

%\begin{figure}[ht]
%\begin{center}
%\includegraphics[height = 9cm, width = 9cm]{pic1.jpeg}
%\caption{A long memory time series\label{ts1}}
%\end{center}
%\end{figure}

%Here's a table.
%\begin{table}[ht]
%\begin{center}
%\begin{tabular}[ht]{|c|lr|c|} 
%%c stands for centre, l for left, r for right; the | puts lines in between, and the hline puts a horizontal line in
%\hline
%$n$ & $\alpha$ &$n\alpha$ & $\beta$\\
%\hline
%1 & 0.2 & 0.2 & 5\\
%\hline
%2 & 0.3 & 0.6 & 4\\
%\hline
%3 & 0.7 & 2.1 & 3\\
%\hline
%\end{tabular}
%\caption{A random table \label{tab1}}
%\end{center}
%\end{table}
%
%\begin{eqnarray}
%y &=& mx + b \label{eq1}\\
%&=& ax+ c
%\label{eq2}
%\end{eqnarray}
%
%This is an un-numbered equation, along with a numbered one. 
%\begin{eqnarray}
%u &=& px \nonumber\\
%p &=& P(X=x) \label{eqn3}
%\end{eqnarray}
%
%Look at Table \ref{tab1} and Figure \ref{ts1} and equations \ref{eq1},  \ref{eq2}, and \ref{eqn3}.
%
%Let's do some matrix algebra now.
%
%\begin{equation}
%det\left(\left|\begin{array}{ccc} 2 & 3 & 5\\
%4 & 4 & 6\\
%9 & 8 & 1
%\end{array}\right|\right) = 42
%\end{equation}
%
%In the equation and eqnarray environments, you don't need to have the dollar sign to enter math mode.
%
%\begin{eqnarray}
%\alpha = \beta_1 \Gamma^{-1}
%\end{eqnarray}
%
%This is citing a reference ~\cite{mygood11111}.  This is citing another ~\cite{mrx05}.  Nobody said something ~\cite{Nobody06}.
