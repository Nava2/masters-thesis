\chapter{Literature Review}

\section{ShelbySim \cite{Tappan2009}}

ShelbySim is an education-oriented software system for designing, simulating, and evaluating computer-engineering based applications. ShelbySim was designed surrounding a new Java-like programming language including a compiler explicitly built around providing extensive diagnostic information such as logging, tracing, and inspection capabilities. These tools provide students with raw data for quantitative analysis, evaluation and reporting of their designs. 

The software is open-source, though not available, and is written using Java allowing full operating system independent support. Additionally, 3D visualized results are provided for viewing developed components. ShelbySim is broken down into three subcomponents: 1) Software component including a custom programming language (Shelby), a compiler, and an interpretation runtime. 

\subsubsection{Talking Points}

\begin{itemize}
\item Most software does not focus on pedagogy
\item Industry software hides underlying information (rightly so for them)
\item Visualization gives "less opaque" view of the system
\item Language 
\end{itemize}

\lipsum


%\begin{figure}[ht]
%\begin{center}
%\includegraphics[height = 9cm, width = 9cm]{pic1.jpeg}
%\caption{A long memory time series\label{ts1}}
%\end{center}
%\end{figure}

Here's a table.
\begin{table}[ht]
\begin{center}
\begin{tabular}[ht]{|c|lr|c|} 
%c stands for centre, l for left, r for right; the | puts lines in between, and the hline puts a horizontal line in
\hline
$n$ & $\alpha$ &$n\alpha$ & $\beta$\\
\hline
1 & 0.2 & 0.2 & 5\\
\hline
2 & 0.3 & 0.6 & 4\\
\hline
3 & 0.7 & 2.1 & 3\\
\hline
\end{tabular}
\caption{A random table \label{tab1}}
\end{center}
\end{table}

\begin{eqnarray}
y &=& mx + b \label{eq1}\\
&=& ax+ c
\label{eq2}
\end{eqnarray}

This is an un-numbered equation, along with a numbered one. 
\begin{eqnarray}
u &=& px \nonumber\\
p &=& P(X=x) \label{eqn3}
\end{eqnarray}

Look at Table \ref{tab1} and Figure \ref{ts1} and equations \ref{eq1},  \ref{eq2}, and \ref{eqn3}.

Let's do some matrix algebra now.

\begin{equation}
det\left(\left|\begin{array}{ccc} 2 & 3 & 5\\
4 & 4 & 6\\
9 & 8 & 1
\end{array}\right|\right) = 42
\end{equation}

In the equation and eqnarray environments, you don't need to have the dollar sign to enter math mode.

\begin{eqnarray}
\alpha = \beta_1 \Gamma^{-1}
\end{eqnarray}

This is citing a reference ~\cite{mygood11111}.  This is citing another ~\cite{mrx05}.  Nobody said something ~\cite{Nobody06}.
